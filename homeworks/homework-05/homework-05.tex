
\documentclass{article}

\usepackage{fancyhdr}
\usepackage{extramarks}
\usepackage{amsmath}
\usepackage{amsthm}
\usepackage{amsfonts}
\usepackage{tikz}
\usepackage[plain]{algorithm}
\usepackage{algpseudocode}

\usetikzlibrary{automata,positioning}

%
% Basic Document Settings
%

\topmargin=-0.45in
\evensidemargin=0in
\oddsidemargin=0in
\textwidth=6.5in
\textheight=9.0in
\headsep=0.25in

\linespread{1.1}

\pagestyle{fancy}
\lhead{\hmwkAuthorName}
\chead{\hmwkClass\ (\hmwkClassInstructor): \hmwkTitle}
\rhead{\firstxmark}
\lfoot{\lastxmark}
\cfoot{\thepage}

\renewcommand\headrulewidth{0.4pt}
\renewcommand\footrulewidth{0.4pt}

\setlength\parindent{0pt}

%
% Create Problem Sections
%

\newcommand{\enterProblemHeader}[1]{
	\nobreak\extramarks{}{Problem \arabic{#1} continued on next page\ldots}\nobreak{}
	\nobreak\extramarks{Problem \arabic{#1} (continued)}{Problem \arabic{#1} continued on next page\ldots}\nobreak{}
}

\newcommand{\exitProblemHeader}[1]{
	\nobreak\extramarks{Problem \arabic{#1} (continued)}{Problem \arabic{#1} continued on next page\ldots}\nobreak{}
	\stepcounter{#1}
	\nobreak\extramarks{Problem \arabic{#1}}{}\nobreak{}
}

\setcounter{secnumdepth}{0}
\newcounter{partCounter}
\newcounter{homeworkProblemCounter}
\setcounter{homeworkProblemCounter}{1}
\nobreak\extramarks{Problem \arabic{homeworkProblemCounter}}{}\nobreak{}

%
% Homework Problem Environment
%
% This environment takes an optional argument. When given, it will adjust the
% problem counter. This is useful for when the problems given for your
% assignment aren't sequential. See the last 3 problems of this template for an
% example.
%
\newenvironment{homeworkProblem}[1][-1]{
	\ifnum#1>0
		\setcounter{homeworkProblemCounter}{#1}
	\fi
	\section{Problem \arabic{homeworkProblemCounter}}
	\setcounter{partCounter}{1}
	\enterProblemHeader{homeworkProblemCounter}
}{
	\exitProblemHeader{homeworkProblemCounter}
}

%
% Homework Details
%   - Title
%   - Due date
%   - Class
%   - Section/Time
%   - Instructor
%   - Author
%

\newcommand{\hmwkTitle}{Problem Set\ \#5}
\newcommand{\hmwkDueDate}{May 29, 2025}
\newcommand{\hmwkClass}{ECON 219}
\newcommand{\hmwkClassInstructor}{Dr. Jasmine N Fuller}
\newcommand{\hmwkAuthorName}{\textbf{Alejandro Ouslan}}

%
% Title Page
%

\title{
	\vspace{2in}
	\textmd{\textbf{\hmwkClass:\ \hmwkTitle}}\\
	\normalsize\vspace{0.1in}\small{Due\ on\ \hmwkDueDate}\\
	\vspace{0.1in}\large{\textit{\hmwkClassInstructor}}
	\vspace{3in}
}

\author{\hmwkAuthorName}
\date{}

\renewcommand{\part}[1]{\textbf{\large Part \Alph{partCounter}}\stepcounter{partCounter}\\}

%
% Various Helper Commands
%

% Useful for algorithms
\newcommand{\alg}[1]{\textsc{\bfseries \footnotesize #1}}

% For derivatives
\newcommand{\deriv}[1]{\frac{\mathrm{d}}{\mathrm{d}x} (#1)}

% For partial derivatives
\newcommand{\pderiv}[2]{\frac{\partial}{\partial #1} (#2)}

% Integral dx
\newcommand{\dx}{\mathrm{d}x}

% Alias for the Solution section header
\newcommand{\solution}{\textbf{\large Solution}}

% Probability commands: Expectation, Variance, Covariance, Bias
\newcommand{\E}{\mathrm{E}}
\newcommand{\Var}{\mathrm{Var}}
\newcommand{\Cov}{\mathrm{Cov}}
\newcommand{\Bias}{\mathrm{Bias}}

\begin{document}

\maketitle

\pagebreak

% Problem 10.3
\begin{homeworkProblem}
	Suppose that a firm's fixed production function is given by
	$$q=\min(5 k,10  l)$$
	\begin{enumerate}
		\item Calculate the firm's long-run total, average, and marginal cost function.

		      \[
			      q = \min(5k, 10l) \Rightarrow 5k = 10l \Rightarrow k = 2l
		      \]

		      \[
			      q = 10l \Rightarrow l = \frac{q}{10}, \quad k = \frac{q}{5}
		      \]

		      \[
			      C(q) = vk + wl = v \cdot \frac{q}{5} + w \cdot \frac{q}{10} = \left( \frac{v}{5} + \frac{w}{10} \right) q
		      \]

		      Therefore:
		      \[
			      \begin{aligned}
				      \text{Total Cost (TC)}    & : C(q) = \left( \frac{v}{5} + \frac{w}{10} \right) q  \\
				      \text{Average Cost (AC)}  & : AC(q) = \frac{C(q)}{q} = \frac{v}{5} + \frac{w}{10} \\
				      \text{Marginal Cost (MC)} & : MC(q) = \frac{dC}{dq} = \frac{v}{5} + \frac{w}{10}
			      \end{aligned}
		      \]
		\item Suppose that $k$ is fixed at 10 in the short run. Calculate the firm's short run total,
		      average, and marginal cost function.
		      \[
			      q = \min(5 \cdot 10, 10l) = \min(50, 10l) \Rightarrow q \leq 50, \quad l \geq \frac{q}{10}
		      \]
		      \[
			      C_{\text{SR}}(q) = 10v + \frac{w}{10}q
		      \]
		      Hence:
		      \[
			      \begin{aligned}
				      \text{Short-Run Total Cost (TC)}    & : C_{\text{SR}}(q) = 10v + \frac{w}{10}q                       \\
				      \text{Short-Run Average Cost (AC)}  & : AC_{\text{SR}}(q) = \frac{10v}{q} + \frac{w}{10}             \\
				      \text{Short-Run Marginal Cost (MC)} & : MC_{\text{SR}}(q) = \frac{dC_{\text{SR}}}{dq} = \frac{w}{10}
			      \end{aligned}
		      \]

		\item Suppose $v=1$ and $w=3$. Calculate this firm's long-run and short-run average and marginal cost curves.
		      \textit{Long Run:}
		      \[
			      C(q) = \left( \frac{1}{5} + \frac{3}{10} \right) q = 0.5q
		      \]
		      \[
			      \begin{aligned}
				      \text{LR TC}(q) & = 0.5q \\
				      \text{LR AC}(q) & = 0.5  \\
				      \text{LR MC}(q) & = 0.5
			      \end{aligned}
		      \]

		      \textit{Short Run:}
		      \[
			      C_{\text{SR}}(q) = 10 \cdot 1 + \frac{3}{10}q = 10 + 0.3q, \quad \text{for } q \leq 50
		      \]
		      \[
			      \begin{aligned}
				      \text{SR TC}(q) & = 10 + 0.3q          \\
				      \text{SR AC}(q) & = \frac{10}{q} + 0.3 \\
				      \text{SR MC}(q) & = 0.3
			      \end{aligned}
		      \]
	\end{enumerate}
\end{homeworkProblem}

% Problem 10.5
\begin{homeworkProblem}
	An enterprising entrepreneur purchases two factories to produce widgets. Each factory produces
	identical products, and each has a production function given by
	$$q=\sqrt{k_{i}l_{i}},i=1,2$$
	The factories differ, however, in the amount of capital equipment each has. In particular, factory $1$ has
	$k_1 =25$, whereas factory $2$ has $k_2 = 100$. Rental rates for $k$ and $l$ are given $w=v=\$1$.
	\begin{enumerate}
		\item If the entrepreneur wishes to minimize short-run total costs of widget production, how should output be allocated between
		      the two factories?

		      \[
			      \min_{l_1, l_2} C = l_1 + l_2
			      \quad \text{s.t.} \quad q = 5\sqrt{l_1} + 10\sqrt{l_2}
		      \]
		      \[
			      \mathcal{L} = l_1 + l_2 + \lambda(q - 5\sqrt{l_1} - 10\sqrt{l_2})
		      \]
		      \[
			      \frac{\partial \mathcal{L}}{\partial l_1} = 1 - \lambda \cdot \frac{5}{2\sqrt{l_1}} = 0
			      \Rightarrow \lambda = \frac{2\sqrt{l_1}}{5}
		      \]
		      \[
			      \frac{\partial \mathcal{L}}{\partial l_2} = 1 - \lambda \cdot \frac{10}{2\sqrt{l_2}} = 0
			      \Rightarrow \lambda = \frac{\sqrt{l_2}}{5}
		      \]
		      \[
			      \frac{2\sqrt{l_1}}{5} = \frac{\sqrt{l_2}}{5} \Rightarrow l_2 = 4l_1
		      \]
		      \[
			      q = 5\sqrt{l_1} + 10\sqrt{l_2} = 5\sqrt{l_1} + 10\sqrt{4l_1} = 25\sqrt{l_1}
			      \Rightarrow \sqrt{l_1} = \frac{q}{25} \Rightarrow l_1 = \left(\frac{q}{25}\right)^2
		      \]
		      \[
			      l_2 = 4l_1 = 4\left(\frac{q}{25}\right)^2
		      \]
		\item Given that output is optimally allocated between the two factories, calculate the short-run total, average, and marginal cost
		      curves. What is the marginal cost of the 100th widget? The 125th widget? The 200th widget?

		      \textbf{Total Cost:}
		      \[
			      C(q) = l_1 + l_2 = \left(\frac{q}{25}\right)^2 + 4\left(\frac{q}{25}\right)^2 = \frac{q^2}{125}
		      \]

		      \textbf{Average Cost:}
		      \[
			      AC(q) = \frac{C(q)}{q} = \frac{q}{125}
		      \]

		      \textbf{Marginal Cost:}
		      \[
			      MC(q) = \frac{dC}{dq} = \frac{2q}{125}
		      \]

		      \textbf{Values:}
		      \[
			      MC(100) = \frac{200}{125} = 1.6
			      \quad
			      MC(125) = \frac{250}{125} = 2
			      \quad
			      MC(200) = \frac{400}{125} = 3.2
		      \]
		\item How should the entrepreneur allocate widget production between the two factories in the long run? Calculate the long-run
		      total, average, and marginal cost curves for widget production
		      In the long run, both capital and labor are variable. Given:
		      \[
			      q_i = \sqrt{k_i l_i} \Rightarrow k_i l_i = q_i^2 \Rightarrow l_i = \frac{q_i^2}{k_i}
			      \Rightarrow C_i = k_i + \frac{q_i^2}{k_i}
		      \]
		      \[
			      \frac{dC_i}{dk_i} = 1 - \frac{q_i^2}{k_i^2} = 0 \Rightarrow k_i = q_i
			      \Rightarrow l_i = q_i
			      \Rightarrow C_i = 2q_i
		      \]
		      \[
			      C(q) = 2q \quad \Rightarrow \quad AC(q) = 2, \quad MC(q) = 2
		      \]
		\item How would your answer to part (c) change if both factories exhibited diminishing returns to scale?

		      \[
			      q_i = (k_i l_i)^{\alpha}, \quad \text{with } 0 < \alpha < \frac{1}{2}
		      \]
		      Then:
		      \begin{itemize}
			      \item Marginal product of inputs declines more rapidly
			      \item Long-run cost curves are increasing and convex
		      \end{itemize}
	\end{enumerate}
\end{homeworkProblem}


% Additional problems 
\begin{homeworkProblem}
	Consider the production function $q=f(k,l) = l^{0.8}k^{0.2}$. The cost of labor and capital are $w$
	and $v$ respectively. Aside from the cost of labor and capital, there are no costs. Initially assume  $k$
	is fixed at $k_1$.
	\begin{enumerate}
		\item Find the short run cost function $C_{SR}(q)$.

		      \[
			      q = l^{0.8}k_1^{0.2} \Rightarrow l = \left(\frac{q}{k_1^{0.2}}\right)^{\frac{1}{0.8}} = q^{1.25}k_1^{-0.25}
		      \]
		      \[
			      C_{SR}(q) = w \cdot l + v \cdot k_1 = w \cdot q^{1.25}k_1^{-0.25} + v \cdot k_1
		      \]
		\item Find short run marginal cost.
		      \[
			      MC_{SR}(q) = \frac{dC_{SR}}{dq} = w \cdot 1.25 \cdot q^{0.25}k_1^{-0.25}
		      \]
		\item Find the short run average cost.
		      \[
			      AC_{SR}(q) = \frac{C_{SR}(q)}{q} = w \cdot q^{0.25}k_1^{-0.25} + \frac{v \cdot k_1}{q}
		      \]
		\item Demostrare that cost is minimized when $AC=MC$.
		      \[
			      AC_{SR}(q) = w \cdot q^{0.25}k_1^{-0.25} + \frac{v \cdot k_1}{q}
		      \]
		      \[
			      MC_{SR}(q) = 1.25 \cdot w \cdot q^{0.25}k_1^{-0.25}
		      \]
		      \[
			      w \cdot q^{0.25}k_1^{-0.25} + \frac{v \cdot k_1}{q} = 1.25 \cdot w \cdot q^{0.25}k_1^{-0.25}
		      \]
		      \[
			      \Rightarrow \frac{v \cdot k_1}{q} = 0.25 \cdot w \cdot q^{0.25}k_1^{-0.25}
		      \]      now asume all inputs are variable.
		\item Find the long run cost function $C_{LR}(q)$.
		      Minimize cost:
		      \[
			      C = w \cdot l + v \cdot k \quad \text{subject to} \quad q = l^{0.8}k^{0.2}
		      \]
		      \[
			      \mathcal{L} = w \cdot l + v \cdot k + \lambda (q - l^{0.8}k^{0.2})
		      \]
		      \[
			      \frac{\partial \mathcal{L}}{\partial l} = w - \lambda \cdot 0.8 \cdot l^{-0.2}k^{0.2} = 0
		      \]
		      \[
			      \frac{\partial \mathcal{L}}{\partial k} = v - \lambda \cdot 0.2 \cdot l^{0.8}k^{-0.8} = 0
		      \]
		      \[
			      \frac{w}{v} = \frac{0.8 \cdot l^{-0.2}k^{0.2}}{0.2 \cdot l^{0.8}k^{-0.8}} = 4 \cdot \frac{k}{l}
			      \Rightarrow k = \frac{w}{4v} \cdot l
		      \]
		      \[
			      q = l^{0.8} \left( \frac{w}{4v}l \right)^{0.2}
			      = \left( \frac{w}{4v} \right)^{0.2} \cdot l
			      \Rightarrow l = \left( \frac{4v}{w} \right)^{0.2} \cdot q
		      \]
		      \[
			      k = \frac{w}{4v} \cdot l = \left( \frac{w}{4v} \right)^{0.8} \cdot q
		      \]
		      \[
			      C_{LR}(q) = w \cdot l + v \cdot k
			      = w \left( \frac{4v}{w} \right)^{0.2} q + v \left( \frac{w}{4v} \right)^{0.8} q = A \cdot q
		      \]
		\item Find the long run marginal Cost.
		      \[
			      MC_{LR}(q) = \frac{dC_{LR}}{dq} = A
		      \]
		\item Find the long run average cost.
		      \[
			      AC_{LR}(q) = \frac{C_{LR}(q)}{q} = A
		      \]

	\end{enumerate}
\end{homeworkProblem}

\end{document}
