
\documentclass{article}

\usepackage{fancyhdr}
\usepackage{extramarks}
\usepackage{amsmath}
\usepackage{amsthm}
\usepackage{amsfonts}
\usepackage{tikz}
\usepackage[plain]{algorithm}
\usepackage{algpseudocode}

\usetikzlibrary{automata,positioning}

%
% Basic Document Settings
%

\topmargin=-0.45in
\evensidemargin=0in
\oddsidemargin=0in
\textwidth=6.5in
\textheight=9.0in
\headsep=0.25in

\linespread{1.1}

\pagestyle{fancy}
\lhead{\hmwkAuthorName}
\chead{\hmwkClass\ (\hmwkClassInstructor): \hmwkTitle}
\rhead{\firstxmark}
\lfoot{\lastxmark}
\cfoot{\thepage}

\renewcommand\headrulewidth{0.4pt}
\renewcommand\footrulewidth{0.4pt}

\setlength\parindent{0pt}

%
% Create Problem Sections
%

\newcommand{\enterProblemHeader}[1]{
	\nobreak\extramarks{}{Problem \arabic{#1} continued on next page\ldots}\nobreak{}
	\nobreak\extramarks{Problem \arabic{#1} (continued)}{Problem \arabic{#1} continued on next page\ldots}\nobreak{}
}

\newcommand{\exitProblemHeader}[1]{
	\nobreak\extramarks{Problem \arabic{#1} (continued)}{Problem \arabic{#1} continued on next page\ldots}\nobreak{}
	\stepcounter{#1}
	\nobreak\extramarks{Problem \arabic{#1}}{}\nobreak{}
}

\setcounter{secnumdepth}{0}
\newcounter{partCounter}
\newcounter{homeworkProblemCounter}
\setcounter{homeworkProblemCounter}{1}
\nobreak\extramarks{Problem \arabic{homeworkProblemCounter}}{}\nobreak{}

%
% Homework Problem Environment
%
% This environment takes an optional argument. When given, it will adjust the
% problem counter. This is useful for when the problems given for your
% assignment aren't sequential. See the last 3 problems of this template for an
% example.
%
\newenvironment{homeworkProblem}[1][-1]{
	\ifnum#1>0
		\setcounter{homeworkProblemCounter}{#1}
	\fi
	\section{Problem \arabic{homeworkProblemCounter}}
	\setcounter{partCounter}{1}
	\enterProblemHeader{homeworkProblemCounter}
}{
	\exitProblemHeader{homeworkProblemCounter}
}

%
% Homework Details
%   - Title
%   - Due date
%   - Class
%   - Section/Time
%   - Instructor
%   - Author
%

\newcommand{\hmwkTitle}{Problem Set\ \#5}
\newcommand{\hmwkDueDate}{May 29, 2025}
\newcommand{\hmwkClass}{ECON 219}
\newcommand{\hmwkClassInstructor}{Dr. Sergio Urzua}
\newcommand{\hmwkAuthorName}{\textbf{Alejandro Ouslan}}

%
% Title Page
%

\title{
	\vspace{2in}
	\textmd{\textbf{\hmwkClass:\ \hmwkTitle}}\\
	\normalsize\vspace{0.1in}\small{Due\ on\ \hmwkDueDate}\\
	\vspace{0.1in}\large{\textit{\hmwkClassInstructor}}
	\vspace{3in}
}

\author{\hmwkAuthorName}
\date{}

\renewcommand{\part}[1]{\textbf{\large Part \Alph{partCounter}}\stepcounter{partCounter}\\}

%
% Various Helper Commands
%

% Useful for algorithms
\newcommand{\alg}[1]{\textsc{\bfseries \footnotesize #1}}

% For derivatives
\newcommand{\deriv}[1]{\frac{\mathrm{d}}{\mathrm{d}x} (#1)}

% For partial derivatives
\newcommand{\pderiv}[2]{\frac{\partial}{\partial #1} (#2)}

% Integral dx
\newcommand{\dx}{\mathrm{d}x}

% Alias for the Solution section header
\newcommand{\solution}{\textbf{\large Solution}}

% Probability commands: Expectation, Variance, Covariance, Bias
\newcommand{\E}{\mathrm{E}}
\newcommand{\Var}{\mathrm{Var}}
\newcommand{\Cov}{\mathrm{Cov}}
\newcommand{\Bias}{\mathrm{Bias}}

\begin{document}

\maketitle

\pagebreak

% Problem 10.3
\begin{homeworkProblem}
  Suppose that a firm's fixed production function is given by
  $$q=\min(5 k,10  l)$$
  \begin{enumerate}
    \item Calculate the firm's long-run total, average, and marginal cost function. 
    \item Suppose that $k$ is fixed at 10 in the short run. Calculate the firm's short run total,
      average, and marginal cost function.
    \item Suppose $v=1$ and $w=3$. Calculate this firm's long-run and short-run average and marginal cost curves.
  \end{enumerate}
\end{homeworkProblem}

% Problem 10.5
\begin{homeworkProblem}
  An enterprising entrepreneur purchases two factories to produce widgets. Each factory produces 
  identical products, and each has a production function given by 
  $$q=\sqrt{k_{i}l_{i}},i=1,2$$
  The factories differ, however, in the amount of capital equipment each has. In particular, factory $1$ has 
  $k_1 =25$, whereas factory $2$ has $k_2 = 100$. Rental rates for $k$ and $l$ are given $w=v=\$1$.
  \begin{enumerate}
    \item If the entrepreneur wishes to minimize short-run total costs of widget production, how should output be allocated between
the two factories?
\item Given that output is optimally allocated between the two factories, calculate the short-run total, average, and marginal cost
curves. What is the marginal cost of the 100th widget? The 125th widget? The 200th widget?
\item How should the entrepreneur allocate widget production between the two factories in the long run? Calculate the long-run
total, average, and marginal cost curves for widget production
\item How would your answer to part (c) change if both factories exhibited diminishing returns to scale?
  \end{enumerate}
\end{homeworkProblem}


% Additional problems 
\begin{homeworkProblem}
  Consider the production function $q=f(k,l) = l^{0.8}k^{0.2}$. The cost of labor and capital are $w$ 
  and $v$ respectively. Aside from the cost of labor and capital, there are no costs. Initially assume  $k$
  is fixed at $k_1$.
  \begin{enumerate}
    \item Find the short run cost function $C_{SR}(q)$.
    \item Find short run marginal cost. 
    \item Find the short run average cost. 
    \item Find short run average cost.
    \item Demostrare that cost is minimized when $AC=MC$.

      now asume all inputs are variable. 
    \item Find the long run cost function $C_{LR}(q)$.
    \item Find the long run marginal Cost.
    \item Find the long run average cost.
  \end{enumerate}
\end{homeworkProblem}

\end{document}
