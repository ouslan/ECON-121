\documentclass{article}

\usepackage{fancyhdr}
\usepackage{extramarks}
\usepackage{amsmath}
\usepackage{amsthm}
\usepackage{amsfonts}
\usepackage{tikz}
\usepackage[plain]{algorithm}
\usepackage{algpseudocode}
\usepackage[inline,shortlabels]{enumitem}
\usetikzlibrary{automata,positioning}

%
% Basic Document Settings
%

\topmargin=-0.45in
\evensidemargin=0in
\oddsidemargin=0in
\textwidth=6.5in
\textheight=9.0in
\headsep=0.25in

\linespread{1.1}

\pagestyle{fancy}
\lhead{\hmwkAuthorName}
\chead{\hmwkClass\ (\hmwkClassInstructor): \hmwkTitle}
\rhead{\firstxmark}
\lfoot{\lastxmark}
\cfoot{\thepage}

\renewcommand\headrulewidth{0.4pt}
\renewcommand\footrulewidth{0.4pt}

\setlength\parindent{0pt}

%
% Create Problem Sections
%

\newcommand{\enterProblemHeader}[1]{
	\nobreak\extramarks{}{Problem \arabic{#1} continued on next page\ldots}\nobreak{}
	\nobreak\extramarks{Problem \arabic{#1} (continued)}{Problem \arabic{#1} continued on next page\ldots}\nobreak{}
}

\newcommand{\exitProblemHeader}[1]{
	\nobreak\extramarks{Problem \arabic{#1} (continued)}{Problem \arabic{#1} continued on next page\ldots}\nobreak{}
	\stepcounter{#1}
	\nobreak\extramarks{Problem \arabic{#1}}{}\nobreak{}
}

\setcounter{secnumdepth}{0}
\newcounter{partCounter}
\newcounter{homeworkProblemCounter}
\setcounter{homeworkProblemCounter}{1}
\nobreak\extramarks{Problem \arabic{homeworkProblemCounter}}{}\nobreak{}

%
% Homework Problem Environment
%
% This environment takes an optional argument. When given, it will adjust the
% problem counter. This is useful for when the problems given for your
% assignment aren't sequential. See the last 3 problems of this template for an
% example.
%
\newenvironment{homeworkProblem}[1][-1]{
	\ifnum#1>0
		\setcounter{homeworkProblemCounter}{#1}
	\fi
	\section{Problem \arabic{homeworkProblemCounter}}
	\setcounter{partCounter}{1}
	\enterProblemHeader{homeworkProblemCounter}
}{
	\exitProblemHeader{homeworkProblemCounter}
}

%
% Homework Details
%   - Title
%   - Due date
%   - Class
%   - Section/Time
%   - Instructor
%   - Author
%

\newcommand{\hmwkTitle}{Problem Set\ \#10}
\newcommand{\hmwkDueDate}{Jun 16, 2025}
\newcommand{\hmwkClass}{ECON 121}
\newcommand{\hmwkClassInstructor}{Dr. Jasmine N Fuller}
\newcommand{\hmwkAuthorName}{\textbf{Alejandro Ouslan}}

%
% Title Page
%

\title{
	\vspace{2in}
	\textmd{\textbf{\hmwkClass:\ \hmwkTitle}}\\
	\normalsize\vspace{0.1in}\small{Due\ on\ \hmwkDueDate}\\
	\vspace{0.1in}\large{\textit{\hmwkClassInstructor}}
	\vspace{3in}
}

\author{\hmwkAuthorName}
\date{}

\renewcommand{\part}[1]{\textbf{\large Part \Alph{partCounter}}\stepcounter{partCounter}\\}

%
% Various Helper Commands
%

% Useful for algorithms
\newcommand{\alg}[1]{\textsc{\bfseries \footnotesize #1}}

% For derivatives
\newcommand{\deriv}[1]{\frac{\mathrm{d}}{\mathrm{d}x} (#1)}

% For partial derivatives
\newcommand{\pderiv}[2]{\frac{\partial}{\partial #1} (#2)}

% Integral dx
\newcommand{\dx}{\mathrm{d}x}

% Alias for the Solution section header
\newcommand{\solution}{\textbf{\large Solution}}

% Probability commands: Expectation, Variance, Covariance, Bias
\newcommand{\E}{\mathrm{E}}
\newcommand{\Var}{\mathrm{Var}}
\newcommand{\Cov}{\mathrm{Cov}}
\newcommand{\Bias}{\mathrm{Bias}}

\begin{document}

\maketitle

\pagebreak

\begin{homeworkProblem}
	Chicken is a classic game regarding brinksmanship; one version is portrayed in the
	movie Rebel Without a Cause, where two teenagers drive directly at one another in
	metal cars, waiting for the other to chicken out and swerve. Consider a game where
	two young rebels/world leaders are hurtling towards each other. The two can be
	Chicken (C) or Macho (M). If both are chicken, we reach the status quo. If one is
	Macho and the other is Chicken, the Macho one “wins” and the Chicken “loses.” If
	both go Macho, a catastrophe occurs. The game can be described in matrix form as
	follows:

	\begin{table}[h!]
		\centering
		\begin{tabular}{c|c c}
			$I P_1/P_2$ & d       & d                   \\
			\hline
			Chicken     & (0, 0)  & (-1, 1)             \\
			Macho       & (1, -1) & (-100000, -1000000) \\
		\end{tabular}
	\end{table}

	\begin{enumerate}[(a)]
		\item Are there any dominated stategies? If so, which ones?
		\item Find any pure stategy NE.
		\item Find the MSNE. What is the expected payoff for Player 1?
	\end{enumerate}
\end{homeworkProblem}

\begin{homeworkProblem}
	Consider the Cournot Fuopoly we studied in Chapter 4, but firm 2 has a marginal cost of
	10 while firm 1 has a marginal cost of 0. As a reminder, demand is $p(q)= 100 - 1$; $q = q_1 + q_2$
	\begin{enumerate}[(a)]
		\item Generte the best response function of firms 1 and 2.
		\item Roughly graph the best responcse correspondence.
		\item Explain the first iteration you would take if you were using IESDS. (just provide one correct answer.)
		\item Find the Nash Equilibrium.
	\end{enumerate}
\end{homeworkProblem}

\end{document}
