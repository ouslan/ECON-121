\documentclass{article}

\usepackage{fancyhdr}
\usepackage{extramarks}
\usepackage{amsmath}
\usepackage{amsthm}
\usepackage{amsfonts}
\usepackage{tikz}
\usepackage[plain]{algorithm}
\usepackage{algpseudocode}

\usetikzlibrary{automata,positioning}

%
% Basic Document Settings
%

\topmargin=-0.45in
\evensidemargin=0in
\oddsidemargin=0in
\textwidth=6.5in
\textheight=9.0in
\headsep=0.25in

\linespread{1.1}

\pagestyle{fancy}
\lhead{\hmwkAuthorName}
\chead{\hmwkClass\ (\hmwkClassInstructor): \hmwkTitle}
\rhead{\firstxmark}
\lfoot{\lastxmark}
\cfoot{\thepage}

\renewcommand\headrulewidth{0.4pt}
\renewcommand\footrulewidth{0.4pt}

\setlength\parindent{0pt}

%
% Create Problem Sections
%

\newcommand{\enterProblemHeader}[1]{
	\nobreak\extramarks{}{Problem \arabic{#1} continued on next page\ldots}\nobreak{}
	\nobreak\extramarks{Problem \arabic{#1} (continued)}{Problem \arabic{#1} continued on next page\ldots}\nobreak{}
}

\newcommand{\exitProblemHeader}[1]{
	\nobreak\extramarks{Problem \arabic{#1} (continued)}{Problem \arabic{#1} continued on next page\ldots}\nobreak{}
	\stepcounter{#1}
	\nobreak\extramarks{Problem \arabic{#1}}{}\nobreak{}
}

\setcounter{secnumdepth}{0}
\newcounter{partCounter}
\newcounter{homeworkProblemCounter}
\setcounter{homeworkProblemCounter}{1}
\nobreak\extramarks{Problem \arabic{homeworkProblemCounter}}{}\nobreak{}

%
% Homework Problem Environment
%
% This environment takes an optional argument. When given, it will adjust the
% problem counter. This is useful for when the problems given for your
% assignment aren't sequential. See the last 3 problems of this template for an
% example.
%
\newenvironment{homeworkProblem}[1][-1]{
	\ifnum#1>0
		\setcounter{homeworkProblemCounter}{#1}
	\fi
	\section{Problem \arabic{homeworkProblemCounter}}
	\setcounter{partCounter}{1}
	\enterProblemHeader{homeworkProblemCounter}
}{
	\exitProblemHeader{homeworkProblemCounter}
}

%
% Homework Details
%   - Title
%   - Due date
%   - Class
%   - Section/Time
%   - Instructor
%   - Author
%

\newcommand{\hmwkTitle}{Problem Set\ \#7}
\newcommand{\hmwkDueDate}{Jun 16, 2025}
\newcommand{\hmwkClass}{ECON 121}
\newcommand{\hmwkClassInstructor}{Dr. Jasmine N Fuller}
\newcommand{\hmwkAuthorName}{\textbf{Alejandro Ouslan}}

%
% Title Page
%

\title{
	\vspace{2in}
	\textmd{\textbf{\hmwkClass:\ \hmwkTitle}}\\
	\normalsize\vspace{0.1in}\small{Due\ on\ \hmwkDueDate}\\
	\vspace{0.1in}\large{\textit{\hmwkClassInstructor}}
	\vspace{3in}
}

\author{\hmwkAuthorName}
\date{}

\renewcommand{\part}[1]{\textbf{\large Part \Alph{partCounter}}\stepcounter{partCounter}\\}

%
% Various Helper Commands
%

% Useful for algorithms
\newcommand{\alg}[1]{\textsc{\bfseries \footnotesize #1}}

% For derivatives
\newcommand{\deriv}[1]{\frac{\mathrm{d}}{\mathrm{d}x} (#1)}

% For partial derivatives
\newcommand{\pderiv}[2]{\frac{\partial}{\partial #1} (#2)}

% Integral dx
\newcommand{\dx}{\mathrm{d}x}

% Alias for the Solution section header
\newcommand{\solution}{\textbf{\large Solution}}

% Probability commands: Expectation, Variance, Covariance, Bias
\newcommand{\E}{\mathrm{E}}
\newcommand{\Var}{\mathrm{Var}}
\newcommand{\Cov}{\mathrm{Cov}}
\newcommand{\Bias}{\mathrm{Bias}}

\begin{document}

\maketitle

\pagebreak

\begin{homeworkProblem}
  A representative consumer has a utility function $U(x,y) = xy$. A representative firm 
  makes good $x$ and has a production function $x = f(k,l) = (kl)^{0.25}$ and an unavoidable 
  fixed cost equal to $A$. There are 100 consumers and, initially, 100 firms. Prices are 
  $w = v =P_y = 1$, and $P_x$ is determined in a competitive market. Representative 
  consumer income is $I= 2$. In the short run, the number of firms is fixed, and capital 
  is fixed at $k_1 = 1$. 
  \begin{enumerate}
    \item Find the representative individual's Marshallian demand for good $x$.
    \item Find the representative firm's supply for good $x$.
    \item Find total demand and total supply for good $x$. 
    \item Solve for the equilibrium price and quantity of good $x$. 
    \item Find total producer surplus in the short run.
  \end{enumerate}
  In the long run, the number of firms is $M$ (determined endogenously), and $k$ is a 
  variable.
  \begin{enumerate}
    \item First, assume $M$ stays at 100, and find equilibrium price and quantity,
      re-doing whatever parts are necessary now that capital isn't fixed.
    \item Suppose $A=2$. Based on economic profit, will $M$ increase or decrease 
      in the long run?
    \item Give $A=1$, find the long run $M$, $X$, and $P_x$. (Hint: re-do long run supply in 
      terms of $M$ instead of 100, find new equilibrium in terms of $M$, and then use the 
      profit condition).
  \end{enumerate}
\end{homeworkProblem}

\begin{homeworkProblem}
  Suppose we are in the long run equilibrium determined by part(h). Note that this long run equilibrium 
  represents a particular short run equilibrium where no firms has an incentive to change $k$, and no firm 
  has an incentive to enter or exit. 
  \begin{enumerate}
    \item Suppose representative consumer income increases form 2 to 4. Find the new short 
      run equilibrium price and quantity where $k$ is fixed at the value determined by part $h$.
    \item Following (i), find the new long run equilibrium price and quantity where $k$ and $M$ are 
      variable. 
    \item Income is 2(so, back to part h). Suppose the government taxes sellers \$1 per unit. 
      what is the new long run equilibrium quantity?
  \end{enumerate}
\end{homeworkProblem
\end{document}

