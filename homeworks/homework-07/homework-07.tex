\documentclass{article}

\usepackage{fancyhdr}
\usepackage{extramarks}
\usepackage{amsmath}
\usepackage{amsthm}
\usepackage{amsfonts}
\usepackage{tikz}
\usepackage[plain]{algorithm}
\usepackage{algpseudocode}

\usetikzlibrary{automata,positioning}

%
% Basic Document Settings
%

\topmargin=-0.45in
\evensidemargin=0in
\oddsidemargin=0in
\textwidth=6.5in
\textheight=9.0in
\headsep=0.25in

\linespread{1.1}

\pagestyle{fancy}
\lhead{\hmwkAuthorName}
\chead{\hmwkClass\ (\hmwkClassInstructor): \hmwkTitle}
\rhead{\firstxmark}
\lfoot{\lastxmark}
\cfoot{\thepage}

\renewcommand\headrulewidth{0.4pt}
\renewcommand\footrulewidth{0.4pt}

\setlength\parindent{0pt}

%
% Create Problem Sections
%

\newcommand{\enterProblemHeader}[1]{
	\nobreak\extramarks{}{Problem \arabic{#1} continued on next page\ldots}\nobreak{}
	\nobreak\extramarks{Problem \arabic{#1} (continued)}{Problem \arabic{#1} continued on next page\ldots}\nobreak{}
}

\newcommand{\exitProblemHeader}[1]{
	\nobreak\extramarks{Problem \arabic{#1} (continued)}{Problem \arabic{#1} continued on next page\ldots}\nobreak{}
	\stepcounter{#1}
	\nobreak\extramarks{Problem \arabic{#1}}{}\nobreak{}
}

\setcounter{secnumdepth}{0}
\newcounter{partCounter}
\newcounter{homeworkProblemCounter}
\setcounter{homeworkProblemCounter}{1}
\nobreak\extramarks{Problem \arabic{homeworkProblemCounter}}{}\nobreak{}

%
% Homework Problem Environment
%
% This environment takes an optional argument. When given, it will adjust the
% problem counter. This is useful for when the problems given for your
% assignment aren't sequential. See the last 3 problems of this template for an
% example.
%
\newenvironment{homeworkProblem}[1][-1]{
	\ifnum#1>0
		\setcounter{homeworkProblemCounter}{#1}
	\fi
	\section{Problem \arabic{homeworkProblemCounter}}
	\setcounter{partCounter}{1}
	\enterProblemHeader{homeworkProblemCounter}
}{
	\exitProblemHeader{homeworkProblemCounter}
}

%
% Homework Details
%   - Title
%   - Due date
%   - Class
%   - Section/Time
%   - Instructor
%   - Author
%

\newcommand{\hmwkTitle}{Problem Set\ \#7}
\newcommand{\hmwkDueDate}{Jun 16, 2025}
\newcommand{\hmwkClass}{ECON 121}
\newcommand{\hmwkClassInstructor}{Dr. Jasmine N Fuller}
\newcommand{\hmwkAuthorName}{\textbf{Alejandro Ouslan}}

%
% Title Page
%

\title{
	\vspace{2in}
	\textmd{\textbf{\hmwkClass:\ \hmwkTitle}}\\
	\normalsize\vspace{0.1in}\small{Due\ on\ \hmwkDueDate}\\
	\vspace{0.1in}\large{\textit{\hmwkClassInstructor}}
	\vspace{3in}
}

\author{\hmwkAuthorName}
\date{}

\renewcommand{\part}[1]{\textbf{\large Part \Alph{partCounter}}\stepcounter{partCounter}\\}

%
% Various Helper Commands
%

% Useful for algorithms
\newcommand{\alg}[1]{\textsc{\bfseries \footnotesize #1}}

% For derivatives
\newcommand{\deriv}[1]{\frac{\mathrm{d}}{\mathrm{d}x} (#1)}

% For partial derivatives
\newcommand{\pderiv}[2]{\frac{\partial}{\partial #1} (#2)}

% Integral dx
\newcommand{\dx}{\mathrm{d}x}

% Alias for the Solution section header
\newcommand{\solution}{\textbf{\large Solution}}

% Probability commands: Expectation, Variance, Covariance, Bias
\newcommand{\E}{\mathrm{E}}
\newcommand{\Var}{\mathrm{Var}}
\newcommand{\Cov}{\mathrm{Cov}}
\newcommand{\Bias}{\mathrm{Bias}}

\begin{document}

\maketitle

\pagebreak

\begin{homeworkProblem}
	A representative consumer has a utility function $U(x,y) = xy$. A representative firm
	makes good $x$ and has a production function $x = f(k,l) = (kl)^{0.25}$ and an unavoidable
	fixed cost equal to $A$. There are 100 consumers and, initially, 100 firms. Prices are
	$w = v =P_y = 1$, and $P_x$ is determined in a competitive market. Representative
	consumer income is $I= 2$. In the short run, the number of firms is fixed, and capital
	is fixed at $k_1 = 1$.
	\begin{enumerate}
		\item Find the representative individual's Marshallian demand for good $x$.

		      \[
			      U(x, y) = x y
		      \]
		      \[
			      P_x x + P_y y = I,
		      \]
		      \[
			      P_x x + y = 2.
		      \]
		      \[
			      \max_{x, y} \; x y \quad \text{subject to} \quad P_x x + y = 2.
		      \]
		      \[
			      \mathcal{L} = x y + \lambda (2 - P_x x - y)
		      \]
		      \[
			      \frac{\partial \mathcal{L}}{\partial x} = y - \lambda P_x = 0 \quad \Rightarrow \quad \lambda = \frac{y}{P_x}
		      \]
		      \[
			      \frac{\partial \mathcal{L}}{\partial y} = x - \lambda = 0 \quad \Rightarrow \quad \lambda = x
		      \]
		      \[
			      x = \frac{y}{P_x} \quad \Rightarrow \quad y = P_x x
		      \]
		      \[
			      P_x x + y = 2 \Rightarrow P_x x + P_x x = 2 \Rightarrow 2 P_x x = 2 \Rightarrow x = \frac{1}{P_x}
		      \]
		      \[
			      x(P_x, I) = \frac{1}{P_x}
		      \]
		\item Find the representative firm's supply for good $x$.

		      \[
			      x = f(k,l) = (k l)^{0.25}.
		      \]
		      \[
			      k = k_1 = 1.
		      \]
		      \[
			      \pi = P_x x - w l - A,
		      \]
		      \[
			      \pi = P_x l^{0.25} - l - A.
		      \]
		      \[
			      \frac{d\pi}{dl} = P_x \cdot 0.25 \, l^{-0.75} - 1 = 0.
		      \]
		      \[
			      0.25 P_x l^{-0.75} = 1 \implies l^{-0.75} = \frac{4}{P_x} \implies l^{0.75} = \frac{P_x}{4}.
		      \]
		      \[
			      l = \left(\frac{P_x}{4}\right)^{\frac{4}{3}}.
		      \]
		      \[
			      x = l^{0.25} = \left[\left(\frac{P_x}{4}\right)^{\frac{4}{3}}\right]^{0.25} = \left(\frac{P_x}{4}\right)^{\frac{1}{3}}.
		      \]
		      \[
			      x(P_x) = \left(\frac{P_x}{4}\right)^{\frac{1}{3}}.
		      \]
		\item Find total demand and total supply for good $x$.
		      \[
			      x(P_x, I) = \frac{1}{P_x}.
		      \]
		      \[
			      X_d = 100 \cdot x(P_x, I) = 100 \cdot \frac{1}{P_x}.
		      \]
		      \[
			      x(P_x) = \left( \frac{P_x}{4} \right)^{\frac{1}{3}}.
		      \]
		      \[
			      X_s = 100 \cdot x(P_x) = 100 \cdot \left( \frac{P_x}{4} \right)^{\frac{1}{3}}.
		      \]
		\item Solve for the equilibrium price and quantity of good $x$.

		      \[
			      X_d = X_s.
		      \]
		      \[
			      X_d = 100 \cdot \frac{1}{P_x}, \quad X_s = 100 \cdot \left( \frac{P_x}{4} \right)^{\frac{1}{3}}.
		      \]
		      \[
			      100 \cdot \frac{1}{P_x} = 100 \cdot \left( \frac{P_x}{4} \right)^{\frac{1}{3}}.
		      \]
		      \[
			      \frac{1}{P_x} = \left( \frac{P_x}{4} \right)^{\frac{1}{3}}.
		      \]
		      \[
			      \left( \frac{1}{P_x} \right)^3 = \frac{P_x}{4}.
		      \]
		      \[
			      \frac{1}{P_x^3} = \frac{P_x}{4}.
		      \]
		      \[
			      1 = \frac{P_x^4}{4},
		      \]
		      \[
			      P_x^4 = 4,
		      \]
		      \[
			      P_x = \sqrt[4]{4} = 2^{1/2} = \sqrt{2}.
		      \]
		      \[
			      P_x = \sqrt{2}.
		      \]
		      \[
			      x = \frac{1}{P_x} = \frac{1}{\sqrt{2}}.
		      \]
		      \[
			      x = \frac{1}{\sqrt{2}}.
		      \]
		\item Find total producer surplus in the short run.
		      \[
			      \pi = P_x x - w l - A.
		      \]
		      \[
			      \pi = P_x \left( \frac{P_x}{4} \right)^{\frac{1}{3}} - l - A.
		      \]
		      \[
			      TR = P_x x = P_x \left( \frac{P_x}{4} \right)^{\frac{1}{3}}.
		      \]
		      \[
			      TVC = w l = l.
		      \]
		      \[
			      TVC = \left( \frac{P_x}{4} \right)^{\frac{4}{3}}.
		      \]
		      \[
			      PS = TR - TVC - A.
		      \]
		      \[
			      PS_{\text{total}} = 100 \cdot PS.
		      \]
	\end{enumerate}
	In the long run, the number of firms is $M$ (determined endogenously), and $k$ is a
	variable.
	\begin{enumerate}
		\item First, assume $M$ stays at 100, and find equilibrium price and quantity,
		      re-doing whatever parts are necessary now that capital isn't fixed.
		      \[
			      x = f(k, l) = (k l)^{0.25}.
		      \]
		      \[
			      \pi = P_x x - w l - A.
		      \]
		      \[
			      \pi = P_x (k l)^{0.25} - w l - A.
		      \]
		      \[
			      \frac{d\pi}{dl} = P_x \cdot 0.25 \cdot (k l)^{-0.75} \cdot k - w = 0.
		      \]
		      \[
			      P_x \cdot 0.25 \cdot k l^{-0.75} = w,
		      \]
		      \[
			      l^{-0.75} = \frac{4 w}{P_x k},
		      \]
		      \[
			      l = \left( \frac{P_x k}{4 w} \right)^{\frac{4}{3}}.
		      \]
		      \[
			      x = (k l)^{0.25}.
		      \]
		      \[
			      x = \left[ k \left( \frac{P_x k}{4 w} \right)^{\frac{4}{3}} \right]^{0.25}.
		      \]
		      \[
			      x = k^{0.25} \cdot \left( \frac{P_x k}{4 w} \right)^{\frac{1}{3}}.
		      \]
		      \[
			      x = k^{0.25} \cdot \left( \frac{P_x}{4 w} \right)^{\frac{1}{3}} \cdot k^{\frac{1}{3}} = k^{\frac{1}{2}} \cdot \left( \frac{P_x}{4 w} \right)^{\frac{1}{3}}.
		      \]
		      \[
			      X_s = 100 \cdot x = 100 \cdot k^{\frac{1}{2}} \cdot \left( \frac{P_x}{4 w} \right)^{\frac{1}{3}}.
		      \]
		      \[
			      X_d = 100 \cdot \frac{1}{P_x}.
		      \]
		      \[
			      X_d = X_s.
		      \]
		      \[
			      100 \cdot \frac{1}{P_x} = 100 \cdot k^{\frac{1}{2}} \cdot \left( \frac{P_x}{4 w} \right)^{\frac{1}{3}}.
		      \]
		      \[
			      \frac{1}{P_x} = k^{\frac{1}{2}} \cdot \left( \frac{P_x}{4 w} \right)^{\frac{1}{3}}.
		      \]
		      \[
			      \left( \frac{1}{P_x} \right)^3 = k^{\frac{3}{2}} \cdot \frac{P_x}{4 w}.
		      \]
		      \[
			      \frac{1}{P_x^3} = k^{\frac{3}{2}} \cdot \frac{P_x}{4 w},
		      \]
		      \[
			      \frac{1}{P_x^4} = \frac{k^{\frac{3}{2}}}{4 w},
		      \]
		      \[
			      P_x^4 = 4 w k^{\frac{3}{2}},
		      \]
		      \[
			      P_x = \sqrt[4]{4 w k^{\frac{3}{2}}}.
		      \]
		      \[
			      P_x = \sqrt[4]{4 w k^{\frac{3}{2}}}.
		      \]
		      \[
			      x = \frac{1}{P_x} = \frac{1}{\sqrt[4]{4 w k^{\frac{3}{2}}}}.
		      \]
		      \[
			      x = \frac{1}{\sqrt[4]{4 w k^{\frac{3}{2}}}}.
		      \]
		\item Suppose $A=2$. Based on economic profit, will $M$ increase or decrease
		      in the long run?
		      \[
			      \pi_{\text{economic}} = \text{Total Revenue} - \text{Total Costs}.
		      \]
		      \[
			      TR = P_x x,
		      \]
		      \[
			      TC = w l + A.
		      \]
		      \[
			      \pi_{\text{economic}} = P_x x - w l - A.
		      \]
		      \[
			      TR = P_x \cdot k^{\frac{1}{2}} \cdot \left( \frac{P_x}{4 w} \right)^{\frac{1}{3}}.
		      \]
		      \[
			      TC = w \cdot \left( \frac{P_x k}{4 w} \right)^{\frac{4}{3}} + A.
		      \]
		      \[
			      \pi_{\text{economic}} = P_x \cdot k^{\frac{1}{2}} \cdot \left( \frac{P_x}{4 w} \right)^{\frac{1}{3}} - w \cdot \left( \frac{P_x k}{4 w} \right)^{\frac{4}{3}} - A.
		      \]
		      \[
			      \pi_{\text{economic}} = 0.
		      \]
		      \[
			      P_x x = w l + A.
		      \]
		\item Give $A=1$, find the long run $M$, $X$, and $P_x$. (Hint: re-do long run supply in
		      terms of $M$ instead of 100, find new equilibrium in terms of $M$, and then use the
		      profit condition).
		      \[
			      x = k^{\frac{1}{2}} \cdot \left( \frac{P_x}{4 w} \right)^{\frac{1}{3}}.
		      \]
		      \[
			      X_s = M \cdot x = M \cdot k^{\frac{1}{2}} \cdot \left( \frac{P_x}{4 w} \right)^{\frac{1}{3}}.
		      \]
		      \[
			      \pi_{\text{economic}} = 0 \quad \Rightarrow \quad P_x x = w l + A.
		      \]
		      \[
			      P_x x = w l + 1.
		      \]
		      \[
			      P_x \cdot 0.25 \cdot k l^{-0.75} = w.
		      \]
		      \[
			      l = \left( \frac{P_x k}{4 w} \right)^{\frac{4}{3}}.
		      \]
		      \[
			      P_x x = w \cdot \left( \frac{P_x k}{4 w} \right)^{\frac{4}{3}} + 1.
		      \]
		      \[
			      P_x \cdot k^{\frac{1}{2}} \cdot \left( \frac{P_x}{4 w} \right)^{\frac{1}{3}} = w \cdot \left( \frac{P_x k}{4 w} \right)^{\frac{4}{3}} + 1.
		      \]
		      \[
			      X_d = 100 \cdot \frac{1}{P_x}.
		      \]
		      \[
			      X_s = X_d \quad \Rightarrow \quad M \cdot k^{\frac{1}{2}} \cdot \left( \frac{P_x}{4 w} \right)^{\frac{1}{3}} = 100 \cdot \frac{1}{P_x}.
		      \]
		      \[
			      M \cdot k^{\frac{1}{2}} \cdot \left( \frac{P_x}{4 w} \right)^{\frac{1}{3}} = 100 \cdot \frac{1}{P_x}.
		      \]
		      \[
			      X_s = M \cdot x.
		      \]
	\end{enumerate}
\end{homeworkProblem}

\begin{homeworkProblem}
	Suppose we are in the long run equilibrium determined by part(h). Note that this long run equilibrium
	represents a particular short run equilibrium where no firms has an incentive to change $k$, and no firm
	has an incentive to enter or exit.
	\begin{enumerate}
		\item Suppose representative consumer income increases form 2 to 4. Find the new short
		      run equilibrium price and quantity where $k$ is fixed at the value determined by part $h$.
		      \[
			      x(P_x, I) = \frac{I}{P_x}.
		      \]
		      \[
			      x(P_x, 4) = \frac{4}{P_x}.
		      \]
		      \[
			      X_d = 100 \cdot \frac{4}{P_x} = \frac{400}{P_x}.
		      \]
		      \[
			      x(P_x) = k^{\frac{1}{2}} \cdot \left( \frac{P_x}{4w} \right)^{\frac{1}{3}}.
		      \]
		      \[
			      X_s = 100 \cdot k^{\frac{1}{2}} \cdot \left( \frac{P_x}{4} \right)^{\frac{1}{3}}.
		      \]
		      \[
			      \frac{400}{P_x} = 100 \cdot k^{\frac{1}{2}} \cdot \left( \frac{P_x}{4} \right)^{\frac{1}{3}}.
		      \]
		      \[
			      \frac{4}{P_x} = k^{\frac{1}{2}} \cdot \left( \frac{P_x}{4} \right)^{\frac{1}{3}}.
		      \]
		      \[
			      \left( \frac{4}{P_x} \right)^3 = k^{\frac{3}{2}} \cdot \frac{P_x}{4}.
		      \]
		      \[
			      \frac{64}{P_x^3} = k^{\frac{3}{2}} \cdot \frac{P_x}{4}.
		      \]
		      \[
			      \frac{64}{P_x^4} = \frac{k^{\frac{3}{2}}}{4},
		      \]
		      \[
			      P_x^4 = 256 \cdot k^{\frac{3}{2}},
		      \]
		      \[
			      P_x = \sqrt[4]{256 \cdot k^{\frac{3}{2}}}.
		      \]
		      \[
			      P_x = \sqrt[4]{256} = 4.
		      \]
		      \[
			      X_d = \frac{400}{4} = 100.
		      \]
		\item Following (i), find the new long run equilibrium price and quantity where $k$ and $M$ are
		      variable.
		      \[
			      x(P_x, 4) = \frac{4}{P_x}.
		      \]
		      \[
			      X_d = 100 \cdot \frac{4}{P_x} = \frac{400}{P_x}.
		      \]
		      \[
			      x(P_x) = k^{\frac{1}{2}} \cdot \left( \frac{P_x}{4w} \right)^{\frac{1}{3}}.
		      \]
		      \[
			      X_s = M \cdot k^{\frac{1}{2}} \cdot \left( \frac{P_x}{4} \right)^{\frac{1}{3}}.
		      \]
		      \[
			      P_x \cdot x = w \cdot l + A.
		      \]
		      \[
			      x = k^{\frac{1}{2}} \cdot \left( \frac{P_x}{4} \right)^{\frac{1}{3}}.
		      \]
		      \[
			      TR = P_x \cdot x = P_x \cdot k^{\frac{1}{2}} \cdot \left( \frac{P_x}{4} \right)^{\frac{1}{3}}.
		      \]
		      \[
			      TC = w \cdot l + A.
		      \]
		      \[
			      P_x \cdot k^{\frac{1}{2}} \cdot \left( \frac{P_x}{4} \right)^{\frac{1}{3}} = w \cdot l + A.
		      \]
		\item Income is 2(so, back to part h). Suppose the government taxes sellers \$1 per unit.
		      what is the new long run equilibrium quantity?

		      A \$1 per unit tax on sellers will affect the cost structure of the firms. The total cost function becomes:
		      \[
			      TC = w \cdot l + A + 1 \cdot x.
		      \]


	\end{enumerate}
\end{homeworkProblem}
\end{document}

