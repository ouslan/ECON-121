

\documentclass{article}

\usepackage{fancyhdr}
\usepackage{extramarks}
\usepackage{amsmath}
\usepackage{amsthm}
\usepackage{amsfonts}
\usepackage{tikz}
\usepackage[plain]{algorithm}
\usepackage{algpseudocode}
\usepackage[inline,shortlabels]{enumitem}
\usetikzlibrary{automata,positioning}
\usepackage{array}
\usepackage{booktabs}
\usepackage{multicol}
\usepackage{amsmath}
\usepackage{graphicx}
\usepackage{extramarks}
%
% Basic Document Settings
%

\topmargin=-0.45in
\evensidemargin=0in
\oddsidemargin=0in
\textwidth=6.5in
\textheight=9.0in
\headsep=0.25in

\linespread{1.1}

\pagestyle{fancy}
\lhead{\hmwkAuthorName}
\chead{\hmwkClass\ (\hmwkClassInstructor): \hmwkTitle}
\rhead{\firstxmark}
\lfoot{\lastxmark}
\cfoot{\thepage}

\renewcommand\headrulewidth{0.4pt}
\renewcommand\footrulewidth{0.4pt}

\setlength\parindent{0pt}

%
% Create Problem Sections
%

\newcommand{\enterProblemHeader}[1]{
	\nobreak\extramarks{}{Problem \arabic{#1} continued on next page\ldots}\nobreak{}
	\nobreak\extramarks{Problem \arabic{#1} (continued)}{Problem \arabic{#1} continued on next page\ldots}\nobreak{}
}

\newcommand{\exitProblemHeader}[1]{
	\nobreak\extramarks{Problem \arabic{#1} (continued)}{Problem \arabic{#1} continued on next page\ldots}\nobreak{}
	\stepcounter{#1}
	\nobreak\extramarks{Problem \arabic{#1}}{}\nobreak{}
}

\setcounter{secnumdepth}{0}
\newcounter{partCounter}
\newcounter{homeworkProblemCounter}
\setcounter{homeworkProblemCounter}{1}
\nobreak\extramarks{Problem \arabic{homeworkProblemCounter}}{}\nobreak{}

%
% Homework Problem Environment
%
% This environment takes an optional argument. When given, it will adjust the
% problem counter. This is useful for when the problems given for your
% assignment aren't sequential. See the last 3 problems of this template for an
% example.
%
\newenvironment{homeworkProblem}[1][-1]{
	\ifnum#1>0
		\setcounter{homeworkProblemCounter}{#1}
	\fi
	\section{Problem \arabic{homeworkProblemCounter}}
	\setcounter{partCounter}{1}
	\enterProblemHeader{homeworkProblemCounter}
}{
	\exitProblemHeader{homeworkProblemCounter}
}

%
% Homework Details
%   - Title
%   - Due date
%   - Class
%   - Section/Time
%   - Instructor
%   - Author
%

\newcommand{\hmwkTitle}{Final}
\newcommand{\hmwkDueDate}{Jun 16, 2025}
\newcommand{\hmwkClass}{ECON 121}
\newcommand{\hmwkClassInstructor}{Dr. Jasmine N Fuller}
\newcommand{\hmwkAuthorName}{\textbf{Alejandro Ouslan}}

%
% Title Page
%

\title{
	\vspace{2in}
	\textmd{\textbf{\hmwkClass:\ \hmwkTitle}}\\
	\normalsize\vspace{0.1in}\small{Due\ on\ \hmwkDueDate}\\
	\vspace{0.1in}\large{\textit{\hmwkClassInstructor}}
	\vspace{3in}
}

\author{\hmwkAuthorName}
\date{}

\renewcommand{\part}[1]{\textbf{\large Part \Alph{partCounter}}\stepcounter{partCounter}\\}

%
% Various Helper Commands
%

% Useful for algorithms
\newcommand{\alg}[1]{\textsc{\bfseries \footnotesize #1}}

% For derivatives
\newcommand{\deriv}[1]{\frac{\mathrm{d}}{\mathrm{d}x} (#1)}

% For partial derivatives
\newcommand{\pderiv}[2]{\frac{\partial}{\partial #1} (#2)}

% Integral dx
\newcommand{\dx}{\mathrm{d}x}

% Alias for the Solution section header
\newcommand{\solution}{\textbf{\large Solution}}

% Probability commands: Expectation, Variance, Covariance, Bias
\newcommand{\E}{\mathrm{E}}
\newcommand{\Var}{\mathrm{Var}}
\newcommand{\Cov}{\mathrm{Cov}}
\newcommand{\Bias}{\mathrm{Bias}}

\begin{document}

\maketitle

\pagebreak

% Problem 1 
\begin{homeworkProblem}
	What is a mechanism? Who is the mechanism designer?

	\textbf{Answer}: A mechanism is a formal system that takes individuals’ private information (like preferences or bids) and produces outcomes,
	such as prices or allocations. The mechanism designer is the planner who creates the rules of the game to achieve a desired objective
	(like efficiency or fairness), even though agents may act strategically.
\end{homeworkProblem}

% Problem 2 
\begin{homeworkProblem}
	Describe the concept of rationalizability.

	\text{Answer:} Rationalize ability refers to strategies that are best responses to some beliefs about what others might do, assuming everyone
	is rational. It captures what rational players could reasonably choose, without assuming they know exactly what others will play.
\end{homeworkProblem}

% Problem 3 
\begin{homeworkProblem}
	What is a dominant strategy?


	\textbf{Answer:} A **dominant strategy** is a strategy that gives a player the highest payoff **no matter what the other players do**.
	In other words, it's always the best choice, regardless of opponents’ actions. If a player has a dominant strategy, they should always play it.

\end{homeworkProblem}

% Problem 4 
\begin{homeworkProblem}
	Solve the following game for its Nash Equilibrium/Equilibria.

	\begin{table}[h!]
		\centering
		\begin{tabular}{l|*{5}{>{\centering\arraybackslash}m{2.5cm}}}
			\toprule
			                  & \textbf{Crime} & \textbf{Action} & \textbf{Thriller} & \textbf{Comedy} & \textbf{Doc.} \\
			\midrule
			\textbf{Crime}    & (10, 10)       & (4, 3)          & (4, 2)            & (4, 1)          & (4, -10)      \\
			\textbf{Action}   & (3, 4)         & (9, 9)          & (3, 2)            & (3, 1)          & (3, -10)      \\
			\textbf{Thriller} & (2, 4)         & (2, 3)          & (8, 8)            & (2, 1)          & (2, -10)      \\
			\textbf{Comedy}   & (1, 4)         & (1, 3)          & (1, 2)            & (5, 5)          & (1, -10)      \\
			\textbf{Doc.}     & (-10, 4)       & (-10, 3)        & (-10, 2)          & (-10, 1)        & (-10, 10)     \\
			\bottomrule
		\end{tabular}
	\end{table}
	\begin{itemize}
		\item \textbf{(Crime, Crime)}: Payoffs (10, 10)
		\item \textbf{(Action, Action)}: Payoffs (9, 9)
		\item \textbf{(Thriller, Thriller)}: Payoffs (8, 8)
		\item \textbf{(Comedy, Comedy)}: Payoffs (5, 5)
	\end{itemize}
\end{homeworkProblem}

% Problem 5 
\begin{homeworkProblem}
	You are in charge of regulating a local utility that runs effectively as a monopoly. The
	utility just recently had a potential entrant, and in response it lowered its prices
	substantially (and temporarily). The potential entrant was scared away before entering at
	all. If, when you ask the utility to lower its prices, it complains its costs are high, why are
	you not convinced?

	\begin{enumerate}
		\item The utility temporarily lowered prices when a competitor appeared, suggesting it can operate at lower prices than it claims.
		\item This behavior resembles predatory pricing—used to deter entry—not a reflection of genuine cost pressures.
		\item As a monopoly, the utility lacks competitive pressure to keep prices aligned with actual costs.
		\item Under cost-plus regulation, it may have incentives to exaggerate costs to justify higher rates.
		\item Overall, its actions undermine the credibility of its high-cost claim and justify regulatory scrutiny.
	\end{enumerate}

\end{homeworkProblem}

% Problem 6 
\begin{homeworkProblem}
	Two firms (1 and 2) compete on price with product differentiation such that demand for
	firm $i$ is $q_i =10 - p_i + p_j$. They incur costs $c_iq_i$, where $c_i$ is “low enough” to encourage
	market participation. All parameters are positive, and costs ($c_1 \and c_2$) are firm dependent.

	\begin{enumerate}[(a)]
		\item Find the prices $p_1 \and p_2$ that occur in equilibrium in the static game.
		      Demand:
		      \[
			      q_1 = 10 - p_1 + p_2, \quad q_2 = 10 - p_2 + p_1
		      \]
		      Profits:
		      \[
			      \pi_1 = (p_1 - c_1)(10 - p_1 + p_2), \quad \pi_2 = (p_2 - c_2)(10 - p_2 + p_1)
		      \]

		      First-order condition (FOC) for Firm 1:
		      \begin{align*}
			      \frac{d\pi_1}{dp_1} & = (10 - p_1 + p_2) + (p_1 - c_1)(-1) = 0 \\
			      \Rightarrow 2p_1    & = 10 + p_2 + c_1                         \\
			      \Rightarrow p_1     & = \frac{10 + p_2 + c_1}{2}
		      \end{align*}

		      FOC for Firm 2:
		      \begin{align*}
			      \frac{d\pi_2}{dp_2} & = (10 - p_2 + p_1) + (p_2 - c_2)(-1) = 0 \\
			      \Rightarrow 2p_2    & = 10 + p_1 + c_2                         \\
			      \Rightarrow p_2     & = \frac{10 + p_1 + c_2}{2}
		      \end{align*}

		      Solving the system:
		      \begin{align*}
			      p_1              & = \frac{10 + \frac{10 + p_1 + c_2}{2} + c_1}{2} \\
			      \Rightarrow 3p_1 & = 30 + c_2 + 2c_1                               \\
			      \Rightarrow p_1  & = \frac{30 + c_2 + 2c_1}{3}
		      \end{align*}
		      \begin{align*}
			      p_2             & = \frac{10 + p_1 + c_2}{2} = \frac{10 + \frac{30 + c_2 + 2c_1}{3} + c_2}{2} \\
			      \Rightarrow p_2 & = \frac{30 + 2c_2 + c_1}{3}
		      \end{align*}

		      \[
			      \boxed{
				      p_1^* = \frac{30 + c_2 + 2c_1}{3}, \quad
				      p_2^* = \frac{30 + 2c_2 + c_1}{3}
			      }
		      \]

		\item For part (b), assume $c_2=30$. Suppose firm 1 invests in a technology that
		      determines $c_1$ paying an investment cost equal to $K-20 c_1$, where K is some
		      constant, and then the firms simultaneously compete on price. Firm 1’s
		      technology investment occurs first, viewed by both firms. What “taxonomy”
		      strategy will firm 1 employ? Why? Hint: What does lowering $c_1$ do to the
		      behavior of firm 2? Find the unique SPNE prices and $c_1$.
		      Substitute $c_2 = 30$ into equilibrium prices:
		      \[
			      p_1 = \frac{60 + 2c_1}{3}, \quad p_2 = \frac{90 + c_1}{3}
		      \]

		      Then quantity for Firm 1:
		      \begin{align*}
			      q_1 & = 10 - p_1 + p_2 = 10 - \frac{60 + 2c_1}{3} + \frac{90 + c_1}{3}
			      = 10 + \frac{30 - c_1}{3} = \frac{60 - c_1}{3}
		      \end{align*}

		      Profit for Firm 1:
		      \begin{align*}
			      p_1 - c_1 & = \frac{60 + 2c_1}{3} - c_1 = \frac{60 - c_1}{3} \\
			      \pi_1     & = (p_1 - c_1)q_1 - (K - 20c_1)
			      = \left( \frac{60 - c_1}{3} \right)^2 - (K - 20c_1)
		      \end{align*}

		      Maximize:
		      \[
			      f(c_1) = \frac{(60 - c_1)^2}{9} + 20c_1
		      \]
		      First derivative:
		      \[
			      f'(c_1) = \frac{2(60 - c_1)(-1)}{9} + 20 = -\frac{2(60 - c_1)}{9} + 20
		      \]

		      Set $f'(c_1) = 0$:
		      \[
			      -\frac{2(60 - c_1)}{9} + 20 = 0 \Rightarrow c_1 = -30 \not\in [0,\infty)
		      \]

		      Since $f'(0) > 0$, profit increases as $c_1$ increases from 0. However, since the maximum occurs at an infeasible $c_1 = -30$, and $f$ is concave downward, the optimum occurs at the boundary: $c_1 = 0$.

		      \[
			      \boxed{
				      c_1^* = 0, \quad
				      p_1 = 20, \quad
				      p_2 = 30
			      }
		      \]

		      Firm 1 captures the entire market:
		      \[
			      q_1 = 10 - 20 + 30 = 20, \quad q_2 = 10 - 30 + 20 = 0
		      \]

		      \textbf{Taxonomy strategy:} Firm 1 commits to an aggressive investment in cost-reduction ($c_1 = 0$), enabling it to price low and capture the entire market. This deters firm 2 from competing effectively. This is a strategic move of:
		      \begin{itemize}
			      \item \textbf{Limit pricing} (pricing to deter competition)
			      \item \textbf{Strategic commitment} to undercut rivals
		      \end{itemize}
	\end{enumerate}
\end{homeworkProblem}

% Problem 7 
\begin{homeworkProblem}
	Consider the following two stage games, the Prisoner’s Dilemma (PD, left) and the
	Revenge game (R, right). When relevant, the discount factor is $\delta , 0<\delta<1$.

	\begin{multicols}{2}

		% First game matrix
		\[
			\begin{array}{c|c|c}
				  & m      & f      \\
				\hline
				M & (4,4)  & (-1,5) \\
				\hline
				F & (5,-1) & (1,1)  \\
			\end{array}
		\]

		% Second game matrix
		\[
			\begin{array}{c|c|c}
				  & l       & g       \\
				\hline
				L & (0,0)   & (-4,-1) \\
				\hline
				G & (-1,-4) & (-3,-3) \\
			\end{array}
		\]

	\end{multicols}

	\begin{enumerate}[(a)]
		\item Take both games separately as static games. Solve for all pure strategy Nash
		      equilibria (NE) for PD, and then do the same for R.

		      \textbf{Prisoner’s Dilemma (PD):}
		      \[
			      \begin{array}{c|c|c}
				        & m      & f      \\
				      \hline
				      M & (4,4)  & (-1,5) \\
				      \hline
				      F & (5,-1) & (1,1)  \\
			      \end{array}
		      \]

		      Player 1:
		      \begin{itemize}
			      \item If Player 2 plays $m$: Player 1 prefers $F$ (5 > 4)
			      \item If Player 2 plays $f$: Player 1 prefers $F$ (1 > -1)
		      \end{itemize}
		      So $F$ is strictly dominant.

		      Player 2:
		      \begin{itemize}
			      \item If Player 1 plays $M$: Player 2 prefers $f$ (5 > 4)
			      \item If Player 1 plays $F$: Player 2 prefers $f$ (1 > -1)
		      \end{itemize}
		      So $f$ is strictly dominant.

		      \textbf{NE:} Unique pure strategy NE is $(F, f)$ with payoff $(1,1)$.

		      \vspace{0.5cm}

		      \textbf{Revenge Game (R):}
		      \[
			      \begin{array}{c|c|c}
				        & l       & g       \\
				      \hline
				      L & (0,0)   & (-4,-1) \\
				      \hline
				      G & (-1,-4) & (-3,-3) \\
			      \end{array}
		      \]

		      Best responses:
		      \begin{itemize}
			      \item If Player 2 plays $l$: Player 1 prefers $L$ (0 > -1)
			      \item If Player 2 plays $g$: Player 1 prefers $G$ (-3 > -4)
			      \item If Player 1 plays $L$: Player 2 prefers $l$ (0 > -1)
			      \item If Player 1 plays $G$: Player 2 prefers $g$ (-3 > -4)
		      \end{itemize}

		      \textbf{NE:} Two pure strategy NE: $(L, l)$ and $(G, g)$

		\item Suppose the players play a three-stage game in the following order: R, PD, PD. Find
		      the two pure strategy subgame perfect Nash equilibria (SPNE).
		      Using backward induction:
		      \begin{itemize}
			      \item In each PD, the unique NE is $(F, f)$ with payoff $(1,1)$.
			      \item So in both PD stages, players will play $(F, f)$.
			      \item In the Revenge game, players can play either NE: $(L, l)$ or $(G, g)$.
		      \end{itemize}

		      So we have two SPNEs:
		      \begin{itemize}
			      \item SPNE 1: Play $(L, l)$ in R, then $(F, f)$ in both PD stages. Payoff: $0 + 1 + 1 = 2$
			      \item SPNE 2: Play $(G, g)$ in R, then $(F, f)$ in both PD stages. Payoff: $-3 + 1 + 1 = -1$
		      \end{itemize}
		\item Suppose the players play a two-stage game in the following order: PD, R. Find a pure
		      strategy SPNE that supports the play of (M,m) in the first round for high enough $\delta$
		      .Find the lowest $\delta$ that supports the SPNE you found in part (c).
		      Goal: Support $(M, m)$ in PD (cooperation). Use trigger strategy:
		      \begin{itemize}
			      \item Stage 1 (PD): Both play $(M, m)$
			      \item Stage 2 (R): Play $(L, l)$ if no deviation in PD; play $(G, g)$ as punishment if any deviation occurs
		      \end{itemize}

		      \textbf{Check incentive constraint:}

		      \begin{itemize}
			      \item No deviation:
			            \[
				            \text{Payoff} = 4 + \delta \cdot 0 = 4
			            \]
			      \item Deviation in PD (e.g., Player 1 plays $F$ while Player 2 plays $m$):
			            \[
				            \text{Payoff} = 5 + \delta \cdot (-3) = 5 - 3\delta
			            \]
			      \item For no incentive to deviate:
			            \[
				            4 \geq 5 - 3\delta \Rightarrow \delta \geq \frac{1}{3}
			            \]
		      \end{itemize}

		      \textbf{Conclusion:} The SPNE is:
		      \begin{itemize}
			      \item Stage 1: $(M, m)$
			      \item Stage 2: $(L, l)$ if no deviation, else $(G, g)$
		      \end{itemize}
		      \textbf{Supported for } $\delta \geq \frac{1}{3}$.
	\end{enumerate}
\end{homeworkProblem}

% Problem 8 
\begin{homeworkProblem}
	Consider a symmetric independent private values setting with two
	buyers. The seller owns an indivisible object which she is commonly
	known to value at zero. Suppose that $v_i \in \{1,2,3\}$ for i = 1,2, and that
	each realization is equally likely (i.e., occurs with the probability 1/3).
	Suppose also that $S_i = {NO, 1, 2, 3}$ for for i = 1,2. That is a bidder
	may choose not to participate or he may bid one of his three possible
	valuations. Finally, suppose that ties in bidding are broken randomly
	without bias.
	\begin{itemize}[(a)]
		\item Suppose the seller announces a second-price sealed bid auction
		      with reserve price of 1. Find a symmetric Bayesian Nash
		      Equilibrium (BNE), $b^\star(v)$, in which the bidders play weakly
		      dominant strategies. How much expected revenue is generated
		      in this BNE?

		      In a second-price auction, truthful bidding is weakly dominant:

		      \[
			      b^\star(v) = v \quad \text{for } v \in \{1,2,3\}
		      \]

		      We compute expected revenue by listing all 9 valuation profiles.

		      \begin{center}
			      \begin{tabular}{ccc|c|c}
				      \( v_1 \) & \( v_2 \) & Winner & 2nd Price & Revenue \\
				      \midrule
				      1         & 1         & Tie    & 1         & 1       \\
				      1         & 2         & 2      & 1         & 1       \\
				      1         & 3         & 3      & 1         & 1       \\
				      2         & 1         & 1      & 1         & 1       \\
				      2         & 2         & Tie    & 2         & 2       \\
				      2         & 3         & 3      & 2         & 2       \\
				      3         & 1         & 1      & 1         & 1       \\
				      3         & 2         & 1      & 2         & 2       \\
				      3         & 3         & Tie    & 3         & 3       \\
			      \end{tabular}
		      \end{center}

		      Each profile has probability \( \frac{1}{9} \). Expected revenue:

		      \[
			      \mathbb{E}[\text{Revenue}] = \frac{1}{9}(1+1+1+1+2+2+1+2+3) = \frac{14}{9} \approx 1.56
		      \]

		\item Suppose the seller announces a second-price sealed bid auction
		      with reserve price of 1. Find an asymmetric BNE, $(b_1(v_1), b_2(v_2))$.

		      Suppose:
		      \[
			      b_1(v_1) = v_1 \quad \text{(truthful)}, \quad
			      b_2(v_2) =
			      \begin{cases}
				      \text{NO} & \text{if } v_2 = 1         \\
				      v_2       & \text{if } v_2 \in \{2,3\}
			      \end{cases}
		      \]

		      Bidder 2 abstains when \( v_2 = 1 \). This is a best response given expected utility is 0 whether he bids or not at \( v = 1 \). This forms an asymmetric BNE.


		\item Suppose the seller announces a first-price sealed-bid auction
		      with reserve price of 1. Find the symmetric BNE, $b^{\star \star}(v)$ which
		      generates the highest expected revenue. How much revenue is
		      generated in the BNE?

		      Assume the symmetric bidding function:

		      \[
			      b^{\star\star}(1) = 1, \quad b^{\star\star}(2) = 1.5, \quad b^{\star\star}(3) = 2
		      \]

		      Compute revenue for each profile (winner pays their bid):

		      \begin{center}
			      \begin{tabular}{ccc|c|c}
				      \( v_1 \) & \( v_2 \) & Winner & Bid & Revenue \\
				      \midrule
				      1         & 1         & Tie    & 1   & 1       \\
				      1         & 2         & 2      & 1.5 & 1.5     \\
				      1         & 3         & 3      & 2   & 2       \\
				      2         & 1         & 1      & 1.5 & 1.5     \\
				      2         & 2         & Tie    & 1.5 & 1.5     \\
				      2         & 3         & 3      & 2   & 2       \\
				      3         & 1         & 1      & 2   & 2       \\
				      3         & 2         & 1      & 2   & 2       \\
				      3         & 3         & Tie    & 2   & 2       \\
			      \end{tabular}
		      \end{center}

		      \[
			      \mathbb{E}[\text{Revenue}] = \frac{1}{9}(1 + 1.5 + 2 + 1.5 + 1.5 + 2 + 2 + 2 + 2) = \frac{15.5}{9} \approx 1.72
		      \]


		\item Suppose the seller announces an all-pay auction with a reserve
		      price of 1. Find a symmetric BNE, $b^{\star \star \star}(v)$. How much expected
		      revenue is generated in this BNE?

		      Assume:

		      \[
			      b^{\star\star\star}(1) = 0, \quad b^{\star\star\star}(2) = 0.25, \quad b^{\star\star\star}(3) = 1
		      \]

		      All bidders pay regardless of winning. Total revenue = sum of both bids.

		      \begin{center}
			      \begin{tabular}{ccc|c}
				      \( v_1 \) & \( v_2 \) & Bids        & Revenue \\
				      \midrule
				      1         & 1         & 0 + 0       & 0       \\
				      1         & 2         & 0 + 0.25    & 0.25    \\
				      1         & 3         & 0 + 1       & 1       \\
				      2         & 1         & 0.25 + 0    & 0.25    \\
				      2         & 2         & 0.25 + 0.25 & 0.5     \\
				      2         & 3         & 0.25 + 1    & 1.25    \\
				      3         & 1         & 1 + 0       & 1       \\
				      3         & 2         & 1 + 0.25    & 1.25    \\
				      3         & 3         & 1 + 1       & 2       \\
			      \end{tabular}
		      \end{center}

		      \[
			      \mathbb{E}[\text{Revenue}] = \frac{1}{9}(0 + 0.25 + 1 + 0.25 + 0.5 + 1.25 + 1 + 1.25 + 2) = \frac{7.5}{9} = \frac{5}{6} \approx 0.83
		      \]

		      \section*{Summary}

		      \begin{center}
			      \begin{tabular}{l|l|c}
				      Auction Type              & Strategy                          & Expected Revenue                  \\
				      \midrule
				      Second-price (symmetric)  & \( b(v) = v \)                    & \( \frac{14}{9} \approx 1.56 \)   \\
				      Second-price (asymmetric) & One abstains at \( v=1 \)         & \( < 1.56 \)                      \\
				      First-price (symmetric)   & \( b(1)=1,\ b(2)=1.5,\ b(3)=2 \)  & \( \frac{15.5}{9} \approx 1.72 \) \\
				      All-pay (symmetric)       & \( b(1)=0,\ b(2)=0.25,\ b(3)=1 \) & \( \frac{5}{6} \approx 0.83 \)    \\
			      \end{tabular}
		      \end{center}

	\end{itemize}
\end{homeworkProblem}

% Problem 9 
\begin{homeworkProblem}
	Let $U(x,y) = \sqrt{x} + \sqrt{y}$
	\begin{enumerate}
		\item Marshallian demand functions for x and y.
		      Maximize
		      \[
			      \max_{x,y} \sqrt{x} + \sqrt{y} \quad \text{s.t.} \quad p_x x + p_y y = I.
		      \]

		      The Lagrangian is
		      \[
			      \mathcal{L} = \sqrt{x} + \sqrt{y} + \lambda (I - p_x x - p_y y).
		      \]

		      First-order conditions:
		      \[
			      \frac{\partial \mathcal{L}}{\partial x} = \frac{1}{2\sqrt{x}} - \lambda p_x = 0 \implies \lambda = \frac{1}{2 p_x \sqrt{x}},
		      \]
		      \[
			      \frac{\partial \mathcal{L}}{\partial y} = \frac{1}{2\sqrt{y}} - \lambda p_y = 0 \implies \lambda = \frac{1}{2 p_y \sqrt{y}}.
		      \]

		      Equate the two expressions for \(\lambda\):
		      \[
			      \frac{1}{2 p_x \sqrt{x}} = \frac{1}{2 p_y \sqrt{y}} \implies p_y \sqrt{y} = p_x \sqrt{x}.
		      \]

		      Square both sides:
		      \[
			      p_y^2 y = p_x^2 x \implies y = \frac{p_x^2}{p_y^2} x.
		      \]

		      Substitute into budget constraint:
		      \[
			      p_x x + p_y y = I \implies p_x x + p_y \frac{p_x^2}{p_y^2} x = I,
		      \]
		      \[
			      p_x x + \frac{p_x^2}{p_y} x = I \implies x \left( p_x + \frac{p_x^2}{p_y} \right) = I,
		      \]
		      \[
			      x = \frac{I}{p_x + \frac{p_x^2}{p_y}} = \frac{I p_y}{p_x (p_x + p_y)}.
		      \]

		      Find \(y\):
		      \[
			      y = \frac{p_x^2}{p_y^2} x = \frac{I p_x}{p_y (p_x + p_y)}.
		      \]

		      \[
			      \boxed{
				      x^* = \frac{I p_y}{p_x (p_x + p_y)}, \quad y^* = \frac{I p_x}{p_y (p_x + p_y)}.
			      }
		      \]
		\item Indirect utility function.
		      \[
			      V(p_x,p_y,I) = \sqrt{x^*} + \sqrt{y^*} = \sqrt{\frac{I p_y}{p_x (p_x + p_y)}} + \sqrt{\frac{I p_x}{p_y (p_x + p_y)}}.
		      \]

		      Factor \(\sqrt{\frac{I}{p_x + p_y}}\):
		      \[
			      = \sqrt{\frac{I}{p_x + p_y}} \left( \sqrt{\frac{p_y}{p_x}} + \sqrt{\frac{p_x}{p_y}} \right) = \sqrt{\frac{I}{p_x + p_y}} \frac{p_x + p_y}{\sqrt{p_x p_y}}.
		      \]

		      Simplify:
		      \[
			      \boxed{
				      V(p_x,p_y,I) = \sqrt{I} \sqrt{\frac{p_x + p_y}{p_x p_y}}.
			      }
		      \]
		\item Compensated (Hicksian) demand functions for x and y.
		      Minimize expenditure subject to utility level \(\bar{U}\):
		      \[
			      \min_{x,y} p_x x + p_y y \quad \text{s.t.} \quad \sqrt{x} + \sqrt{y} = \bar{U}.
		      \]

		      Lagrangian:
		      \[
			      \mathcal{L} = p_x x + p_y y + \mu (\bar{U} - \sqrt{x} - \sqrt{y}).
		      \]

		      FOCs:
		      \[
			      \frac{\partial \mathcal{L}}{\partial x} = p_x - \frac{\mu}{2\sqrt{x}} = 0 \implies \mu = 2 p_x \sqrt{x},
		      \]
		      \[
			      \frac{\partial \mathcal{L}}{\partial y} = p_y - \frac{\mu}{2\sqrt{y}} = 0 \implies \mu = 2 p_y \sqrt{y}.
		      \]

		      Equate:
		      \[
			      2 p_x \sqrt{x} = 2 p_y \sqrt{y} \implies p_x \sqrt{x} = p_y \sqrt{y}.
		      \]

		      Square both sides:
		      \[
			      p_x^2 x = p_y^2 y \implies y = \frac{p_x^2}{p_y^2} x.
		      \]

		      Use utility constraint:
		      \[
			      \bar{U} = \sqrt{x} + \sqrt{y} = \sqrt{x} + \frac{p_x}{p_y} \sqrt{x} = \sqrt{x} \frac{p_x + p_y}{p_y},
		      \]
		      \[
			      \implies \sqrt{x} = \bar{U} \frac{p_y}{p_x + p_y} \implies x = \bar{U}^2 \frac{p_y^2}{(p_x + p_y)^2}.
		      \]

		      Similarly,
		      \[
			      y = \bar{U}^2 \frac{p_x^2}{(p_x + p_y)^2}.
		      \]

		      \[
			      \boxed{
				      h_x(p_x,p_y,\bar{U}) = \bar{U}^2 \frac{p_y^2}{(p_x + p_y)^2}, \quad
				      h_y(p_x,p_y,\bar{U}) = \bar{U}^2 \frac{p_x^2}{(p_x + p_y)^2}.
			      }
		      \]
		\item Expenditure function.
		      Substitute Hicksian demands:
		      \[
			      e(p_x,p_y,\bar{U}) = p_x h_x + p_y h_y = \bar{U}^2 \frac{p_x p_y^2 + p_y p_x^2}{(p_x + p_y)^2} = \bar{U}^2 \frac{p_x p_y (p_x + p_y)}{(p_x + p_y)^2} = \bar{U}^2 \frac{p_x p_y}{p_x + p_y}.
		      \]

		      \[
			      \boxed{
				      e(p_x,p_y,\bar{U}) = \bar{U}^2 \frac{p_x p_y}{p_x + p_y}.
			      }
		      \]
		\item Write the Slutsky equation for good x.
		      The Slutsky equation decomposes the effect of a price change on demand:
		      \[
			      \frac{\partial x^*}{\partial p_x} = \frac{\partial h_x}{\partial p_x} - \frac{\partial x^*}{\partial I} x^*.
		      \]

		      Where:
		      \begin{itemize}
			      \item \(\frac{\partial x^*}{\partial p_x}\) is the total effect of the price change,
			      \item \(\frac{\partial h_x}{\partial p_x}\) is the substitution effect (holding utility constant),
			      \item \(\frac{\partial x^*}{\partial I} x^*\) is the income effect.
		      \end{itemize}
		\item Explain, in words, how you could have derived Marshallian demand in a different
		      way.

		      Instead of using the Lagrangian multiplier method, one could solve the budget constraint for \(y\), substitute into the utility function, and maximize the resulting single-variable function in \(x\). Alternatively, use Roy's identity applied to the indirect utility function.

		\item Explain, in words, how you could have derived Hicksian demand in a different
		      way.
		      Instead of minimizing expenditure directly, one could use Roy's identity to derive Hicksian demands from the indirect utility function or invert the Marshallian demand functions to solve for Hicksian demands.
	\end{enumerate}
\end{homeworkProblem}

% Problem 10
\begin{homeworkProblem}
	A seller sells a good to a prospective buyers. The buyer values the good at $\theta q$, Where $\theta$ is
	his (privately known) marginal utility of quality and $q$ is the good's quality. It is common
	knowledge that $\theta$ is high ($\theta = 2$) with a probability of $\frac{1}{4}$ and $\theta$ is low $(\theta = 1)$ with
	probability of $\frac{3}{4}$. The monopolist incurres a cost based on quality $c(q) = \frac{q}{2}$ so that his
	profits is $q - \frac{q^2}{2}$. The buyer can reject an offer (not buy anything) and get a payoff of 0, or
	he can buy a good and get a payoff of $\theta q - p$. The seller offer a menu consisting of
	$\{p_1, q_1, p_2, q_2 \}$, where the subscript means the price and quality is meant for the seller of
	that type ($\theta$), and the buyer picks which good she wants.


	\begin{enumerate}[(a)]
		\item Suppose, for part (a) only, the seller observe $\theta$ directly and can offer a single type
		      of good based on the buyer's type $\{p_i, q_i \}$, $i\in [1,2]$. Construct the optimal
		      price/quality combination when $\theta=1$ and when $\theta=2$.
		      \textbf{Case 1: When \( \theta = 1 \)}

		      For a buyer with \( \theta = 1 \), their utility is:
		      \[
			      U_1 = q - p
		      \]
		      To ensure that the buyer is willing to buy, the seller needs to guarantee that:
		      \[
			      U_1 \geq 0 \quad \Rightarrow \quad q - p \geq 0 \quad \Rightarrow \quad p \leq q
		      \]
		      The monopolist wants to choose \( p \) and \( q \) to maximize their profit, given by:
		      \[
			      \pi = p - \frac{q}{2}
		      \]
		      We now solve for the optimal \( p \) and \( q \). To maximize profit, we substitute \( p = q \) into the profit function:
		      \[
			      \pi = q - \frac{q}{2} = \frac{q}{2}
		      \]
		      To maximize \( \frac{q}{2} \), we differentiate with respect to \( q \):
		      \[
			      \frac{d\pi}{dq} = \frac{1}{2} \quad \Rightarrow \quad \text{Max profit occurs when} \quad q = 2
		      \]
		      Thus, when \( \theta = 1 \), the seller should offer:
		      \[
			      q_1 = 2 \quad \text{and} \quad p_1 = 2
		      \]

		      \textbf{Case 2: When \( \theta = 2 \)}

		      For a buyer with \( \theta = 2 \), their utility is:
		      \[
			      U_2 = 2q - p
		      \]
		      The seller needs to ensure that:
		      \[
			      U_2 \geq 0 \quad \Rightarrow \quad 2q - p \geq 0 \quad \Rightarrow \quad p \leq 2q
		      \]
		      The monopolist wants to choose \( p \) and \( q \) to maximize their profit, given by:
		      \[
			      \pi = p - \frac{q}{2}
		      \]
		      We substitute \( p = 2q \) into the profit function:
		      \[
			      \pi = 2q - \frac{q}{2} = \frac{3q}{2}
		      \]
		      To maximize \( \frac{3q}{2} \), we differentiate with respect to \( q \):
		      \[
			      \frac{d\pi}{dq} = \frac{3}{2} \quad \Rightarrow \quad \text{Max profit occurs when} \quad q = 4
		      \]
		      Thus, when \( \theta = 2 \), the seller should offer:
		      \[
			      q_2 = 4 \quad \text{and} \quad p_2 = 8
		      \]
		\item Suppose the seller ask the buyer what hist type is, assume he answers honestly, and offer
		      and contract like in part (a). Who will lie, and why?
		      If the seller offers a menu like in part (a), the buyer may be tempted to lie about their type.

		      - If a buyer with \( \theta = 1 \) lies and claims to be of type \( \theta = 2 \), they will be offered \( q_2 = 4 \) and \( p_2 = 8 \). The utility for this buyer is:
		      \[
			      U_1 = 2(4) - 8 = 8 - 8 = 0
		      \]
		      This utility is the same as the utility from rejecting the offer (0), so the buyer would not buy the good. Thus, lying does not benefit them.

		      - A buyer with \( \theta = 2 \) who lies and claims to be of type \( \theta = 1 \) would be offered \( q_1 = 2 \) and \( p_1 = 2 \). The utility for this buyer is:
		      \[
			      U_2 = 1(2) - 2 = 2 - 2 = 0
		      \]
		      Again, this is the same as the utility from rejecting the offer, so this buyer would not benefit from lying either.

		      \textbf{Conclusion:} In this setup, \textit{no one will lie} because the utility from lying does not exceed the utility from rejecting the offer.

		\item Construct the optimal contract where everyone buys their appropriate good.
		      The optimal contract would ensure that each buyer purchases the good designed for their type. The seller offers the following:

		      For \( \theta = 1 \), the seller offers:
		      \[
			      q_1 = 2 \quad \text{and} \quad p_1 = 2
		      \]
		      For \( \theta = 2 \), the seller offers:
		      \[
			      q_2 = 4 \quad \text{and} \quad p_2 = 8
		      \]

		      Each buyer will purchase the good that maximizes their utility:
		      - A buyer with \( \theta = 1 \) will choose \( q_1 = 2 \) and \( p_1 = 2 \).
		      - A buyer with \( \theta = 2 \) will choose \( q_2 = 4 \) and \( p_2 = 8 \).

		      Thus, everyone buys the appropriate good.

		\item Whose good has an inefficient level of quality? Is it too high or low? Why does
		      the seller do this?

		      The monopolist is offering two different qualities for two different types of buyers. The question is whether either of these levels of quality is inefficient.

		      - For \( \theta = 1 \), the buyer gets \( q_1 = 2 \), which is optimal given the profit maximization condition.
		      - For \( \theta = 2 \), the buyer gets \( q_2 = 4 \), which is also optimal.

		      The monopolist may be offering a quality that is too high for the buyer with \( \theta = 1 \) and too low for the buyer with \( \theta = 2 \). The monopolist offers higher quality to match the buyer's willingness to pay, but the higher quality could be considered inefficient from a social perspective.

		      The seller does this to maximize profit. Higher quality increases the buyer's willingness to pay, which allows the monopolist to extract more surplus from the buyer. Thus, the seller offers higher quality than what would be socially optimal to increase their profit.

	\end{enumerate}
\end{homeworkProblem}
\end{document}
