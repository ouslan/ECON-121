

\documentclass{article}

\usepackage{fancyhdr}
\usepackage{extramarks}
\usepackage{amsmath}
\usepackage{amsthm}
\usepackage{amsfonts}
\usepackage{tikz}
\usepackage[plain]{algorithm}
\usepackage{algpseudocode}
\usepackage[inline,shortlabels]{enumitem}
\usetikzlibrary{automata,positioning}
\usepackage{array}
\usepackage{booktabs}
\usepackage{multicol}
\usepackage{amsmath}
\usepackage{graphicx}
\usepackage{extramarks}
%
% Basic Document Settings
%

\topmargin=-0.45in
\evensidemargin=0in
\oddsidemargin=0in
\textwidth=6.5in
\textheight=9.0in
\headsep=0.25in

\linespread{1.1}

\pagestyle{fancy}
\lhead{\hmwkAuthorName}
\chead{\hmwkClass\ (\hmwkClassInstructor): \hmwkTitle}
\rhead{\firstxmark}
\lfoot{\lastxmark}
\cfoot{\thepage}

\renewcommand\headrulewidth{0.4pt}
\renewcommand\footrulewidth{0.4pt}

\setlength\parindent{0pt}

%
% Create Problem Sections
%

\newcommand{\enterProblemHeader}[1]{
	\nobreak\extramarks{}{Problem \arabic{#1} continued on next page\ldots}\nobreak{}
	\nobreak\extramarks{Problem \arabic{#1} (continued)}{Problem \arabic{#1} continued on next page\ldots}\nobreak{}
}

\newcommand{\exitProblemHeader}[1]{
	\nobreak\extramarks{Problem \arabic{#1} (continued)}{Problem \arabic{#1} continued on next page\ldots}\nobreak{}
	\stepcounter{#1}
	\nobreak\extramarks{Problem \arabic{#1}}{}\nobreak{}
}

\setcounter{secnumdepth}{0}
\newcounter{partCounter}
\newcounter{homeworkProblemCounter}
\setcounter{homeworkProblemCounter}{1}
\nobreak\extramarks{Problem \arabic{homeworkProblemCounter}}{}\nobreak{}

%
% Homework Problem Environment
%
% This environment takes an optional argument. When given, it will adjust the
% problem counter. This is useful for when the problems given for your
% assignment aren't sequential. See the last 3 problems of this template for an
% example.
%
\newenvironment{homeworkProblem}[1][-1]{
	\ifnum#1>0
		\setcounter{homeworkProblemCounter}{#1}
	\fi
	\section{Problem \arabic{homeworkProblemCounter}}
	\setcounter{partCounter}{1}
	\enterProblemHeader{homeworkProblemCounter}
}{
	\exitProblemHeader{homeworkProblemCounter}
}

%
% Homework Details
%   - Title
%   - Due date
%   - Class
%   - Section/Time
%   - Instructor
%   - Author
%

\newcommand{\hmwkTitle}{Final}
\newcommand{\hmwkDueDate}{Jun 16, 2025}
\newcommand{\hmwkClass}{ECON 121}
\newcommand{\hmwkClassInstructor}{Dr. Jasmine N Fuller}
\newcommand{\hmwkAuthorName}{\textbf{Alejandro Ouslan}}

%
% Title Page
%

\title{
	\vspace{2in}
	\textmd{\textbf{\hmwkClass:\ \hmwkTitle}}\\
	\normalsize\vspace{0.1in}\small{Due\ on\ \hmwkDueDate}\\
	\vspace{0.1in}\large{\textit{\hmwkClassInstructor}}
	\vspace{3in}
}

\author{\hmwkAuthorName}
\date{}

\renewcommand{\part}[1]{\textbf{\large Part \Alph{partCounter}}\stepcounter{partCounter}\\}

%
% Various Helper Commands
%

% Useful for algorithms
\newcommand{\alg}[1]{\textsc{\bfseries \footnotesize #1}}

% For derivatives
\newcommand{\deriv}[1]{\frac{\mathrm{d}}{\mathrm{d}x} (#1)}

% For partial derivatives
\newcommand{\pderiv}[2]{\frac{\partial}{\partial #1} (#2)}

% Integral dx
\newcommand{\dx}{\mathrm{d}x}

% Alias for the Solution section header
\newcommand{\solution}{\textbf{\large Solution}}

% Probability commands: Expectation, Variance, Covariance, Bias
\newcommand{\E}{\mathrm{E}}
\newcommand{\Var}{\mathrm{Var}}
\newcommand{\Cov}{\mathrm{Cov}}
\newcommand{\Bias}{\mathrm{Bias}}

\begin{document}

\maketitle

\pagebreak

% Problem 1 
\begin{homeworkProblem}
	What is a mechanism? Who is the mechanism designer?
\end{homeworkProblem}

% Problem 2 
\begin{homeworkProblem}
	Describe the concept of rationalizability.
\end{homeworkProblem}

% Problem 3 
\begin{homeworkProblem}
	What is a dominant strategy?
\end{homeworkProblem}

% Problem 4 
\begin{homeworkProblem}
	Solve the following game for its Nash Equilibrium/Equilibria.

	\begin{table}[h!]
		\centering
		\begin{tabular}{l|*{5}{>{\centering\arraybackslash}m{2.5cm}}}
			\toprule
			                  & \textbf{Crime} & \textbf{Action} & \textbf{Thriller} & \textbf{Comedy} & \textbf{Doc.} \\
			\midrule
			\textbf{Crime}    & (10, 10)       & (4, 3)          & (4, 2)            & (4, 1)          & (4, -10)      \\
			\textbf{Action}   & (3, 4)         & (9, 9)          & (3, 2)            & (3, 1)          & (3, -10)      \\
			\textbf{Thriller} & (2, 4)         & (2, 3)          & (8, 8)            & (2, 1)          & (2, -10)      \\
			\textbf{Comedy}   & (1, 4)         & (1, 3)          & (1, 2)            & (5, 5)          & (1, -10)      \\
			\textbf{Doc.}     & (-10, 4)       & (-10, 3)        & (-10, 2)          & (-10, 1)        & (-10, 10)     \\
			\bottomrule
		\end{tabular}
	\end{table}

\end{homeworkProblem}

% Problem 5 
\begin{homeworkProblem}
	You are in charge of regulating a local utility that runs effectively as a monopoly. The
	utility just recently had a potential entrant, and in response it lowered its prices
	substantially (and temporarily). The potential entrant was scared away before entering at
	all. If, when you ask the utility to lower its prices, it complains its costs are high, why are
	you not convinced?
\end{homeworkProblem}

% Problem 6 
\begin{homeworkProblem}
	Two firms (1 and 2) compete on price with product differentiation such that demand for
	firm $i$ is $q_i =10 - p_i + p_j$. They incur costs $c_iq_i$, where $c_i$ is “low enough” to encourage
	market participation. All parameters are positive, and costs ($c_1 \and c_2$) are firm dependent.

	\begin{enumerate}[(a)]
		\item Find the prices $p_1 \and p_2$ that occur in equilibrium in the static game.
		\item For part (b), assume $c_2=30$. Suppose firm 1 invests in a technology that
		      determines $c_1$ paying an investment cost equal to $K-20 c_1$, where K is some
		      constant, and then the firms simultaneously compete on price. Firm 1’s
		      technology investment occurs first, viewed by both firms. What “taxonomy”
		      strategy will firm 1 employ? Why? Hint: What does lowering $c_1$ do to the
		      behavior of firm 2? Find the unique SPNE prices and $c_1$.
		\item
	\end{enumerate}
\end{homeworkProblem}

% Problem 7 
\begin{homeworkProblem}
	Consider the following two stage games, the Prisoner’s Dilemma (PD, left) and the
	Revenge game (R, right). When relevant, the discount factor is $\delta , 0<\delta<1$.

	\begin{multicols}{2}

		% First game matrix
		\[
			\begin{array}{c|c|c}
				  & m      & f      \\
				\hline
				M & (4,4)  & (-1,5) \\
				\hline
				F & (5,-1) & (1,1)  \\
			\end{array}
		\]

		% Second game matrix
		\[
			\begin{array}{c|c|c}
				  & l       & g       \\
				\hline
				L & (0,0)   & (-4,-1) \\
				\hline
				G & (-1,-4) & (-3,-3) \\
			\end{array}
		\]

	\end{multicols}

	\begin{enumerate}[(a)]
		\item Take both games separately as static games. Solve for all pure strategy Nash
		      equilibria (NE) for PD, and then do the same for R.
		\item Suppose the players play a three-stage game in the following order: R, PD, PD. Find
		      the two pure strategy subgame perfect Nash equilibria (SPNE).
		\item Suppose the players play a two-stage game in the following order: PD, R. Find a pure
		      strategy SPNE that supports the play of (M,m) in the first round for high enough $\delta$
		      .Find the lowest $\delta$ that supports the SPNE you found in part (c).
	\end{enumerate}
\end{homeworkProblem}

% Problem 8 
\begin{homeworkProblem}
	Consider a symmetric independent private values setting with two
	buyers. The seller owns an indivisible object which she is commonly
	known to value at zero. Suppose that $v_i \in \{1,2,3\}$ for i = 1,2, and that
	each realization is equally likely (i.e., occurs with the probability 1/3).
	Suppose also that $S_i = {NO, 1, 2, 3}$ for for i = 1,2. That is a bidder
	may choose not to participate or he may bid one of his three possible
	valuations. Finally, suppose that ties in bidding are broken randomly
	without bias.
	\begin{itemize}[(a)]
		\item Suppose the seller announces a second-price sealed bid auction
		      with reserve price of 1. Find a symmetric Bayesian Nash
		      Equilibrium (BNE), $b^\star(v)$, in which the bidders play weakly
		      dominant strategies. How much expected revenue is generated
		      in this BNE?
		\item Suppose the seller announces a second-price sealed bid auction
		      with reserve price of 1. Find an asymmetric BNE, $(b_1(v_1), b_2(v_2))$.
		\item Suppose the seller announces a first-price sealed-bid auction
		      with reserve price of 1. Find the symmetric BNE, $b^{\star \star}(v)$ which
		      generates the highest expected revenue. How much revenue is
		      generated in the BNE?
		\item Suppose the seller announces an all-pay auction with a reserve
		      price of 1. Find a symmetric BNE, $b^{\star \star \star}(v)$. How much expected
		      revenue is generated in this BNE?
	\end{itemize}
\end{homeworkProblem}

% Problem 9 
\begin{homeworkProblem}
	Let $U(x,y) = \sqrt{x} + \sqrt{y}$
	\begin{enumerate}
		\item Marshallian demand functions for x and y.
		\item Indirect utility function.
		\item Compensated (Hicksian) demand functions for x and y.
		\item Expenditure function.
		\item Write the Slutsky equation for good x.
		\item Explain, in words, how you could have derived Marshallian demand in a different
		      way.
		\item Explain, in words, how you could have derived Hicksian demand in a different
		      way.
	\end{enumerate}
\end{homeworkProblem}

% Problem 10
\begin{homeworkProblem}
	A seller sells a good to a prospective buyers. The buyer values the good at $\theta q$, Where $\theta$ is
	his (privately known) marginal utility of quality and $q$ is the good's quality. It is common
	knowledge that $\theta$ is high ($\theta = 2$) with a probability of $\frac{1}{4}$ and $\theta$ is low $(\theta = 1)$ with
	probability of $\frac{3}{4}$. The monopolist incurres a cost based on quality $c(q) = \frac{q}{2}$ so that his
	profits is $q - \frac{q^2}{2}$. The buyer can reject an offer (not buy anything) and get a payoff of 0, or
	he can buy a good and get a payoff of $\theta q - p$. The seller offer a menu consisting of
	$\{p_1, q_1, p_2, q_2 \}$, where the subscript means the price and quality is meant for the seller of
	that type ($\theta$), and the buyer picks which good she wants.


	\begin{enumerate}[(a)]
		\item Suppose, for part (a) only, the seller observe $\theta$ directly and can offer a single type
		      of good based on the buyer's type $\{p_i, q_i \}$, $i\in [1,2]$. Construct the optimal
		      price/quality combination when $\theta=1$ and when $\theta=2$.
		\item Suppose the seller ask the buyer what hist type is, assume he answers honestly, and offer
		      and contract like in part (a). Who will lie, and why?
		\item Construct the optimal contract where everyone buys their appropriate good.
		\item Whose good has an inefficient level of quality? Is it too high or low? Why does
		      the seller do this?
	\end{enumerate}
\end{homeworkProblem}
\end{document}
