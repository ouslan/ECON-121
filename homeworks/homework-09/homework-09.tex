\documentclass{article}

\usepackage{fancyhdr}
\usepackage{extramarks}
\usepackage{amsmath}
\usepackage{amsthm}
\usepackage{amsfonts}
\usepackage{tikz}
\usepackage[plain]{algorithm}
\usepackage{algpseudocode}

\usetikzlibrary{automata,positioning}

%
% Basic Document Settings
%

\topmargin=-0.45in
\evensidemargin=0in
\oddsidemargin=0in
\textwidth=6.5in
\textheight=9.0in
\headsep=0.25in

\linespread{1.1}

\pagestyle{fancy}
\lhead{\hmwkAuthorName}
\chead{\hmwkClass\ (\hmwkClassInstructor): \hmwkTitle}
\rhead{\firstxmark}
\lfoot{\lastxmark}
\cfoot{\thepage}

\renewcommand\headrulewidth{0.4pt}
\renewcommand\footrulewidth{0.4pt}

\setlength\parindent{0pt}

%
% Create Problem Sections
%

\newcommand{\enterProblemHeader}[1]{
	\nobreak\extramarks{}{Problem \arabic{#1} continued on next page\ldots}\nobreak{}
	\nobreak\extramarks{Problem \arabic{#1} (continued)}{Problem \arabic{#1} continued on next page\ldots}\nobreak{}
}

\newcommand{\exitProblemHeader}[1]{
	\nobreak\extramarks{Problem \arabic{#1} (continued)}{Problem \arabic{#1} continued on next page\ldots}\nobreak{}
	\stepcounter{#1}
	\nobreak\extramarks{Problem \arabic{#1}}{}\nobreak{}
}

\setcounter{secnumdepth}{0}
\newcounter{partCounter}
\newcounter{homeworkProblemCounter}
\setcounter{homeworkProblemCounter}{1}
\nobreak\extramarks{Problem \arabic{homeworkProblemCounter}}{}\nobreak{}

%
% Homework Problem Environment
%
% This environment takes an optional argument. When given, it will adjust the
% problem counter. This is useful for when the problems given for your
% assignment aren't sequential. See the last 3 problems of this template for an
% example.
%
\newenvironment{homeworkProblem}[1][-1]{
	\ifnum#1>0
		\setcounter{homeworkProblemCounter}{#1}
	\fi
	\section{Problem \arabic{homeworkProblemCounter}}
	\setcounter{partCounter}{1}
	\enterProblemHeader{homeworkProblemCounter}
}{
	\exitProblemHeader{homeworkProblemCounter}
}

%
% Homework Details
%   - Title
%   - Due date
%   - Class
%   - Section/Time
%   - Instructor
%   - Author
%

\newcommand{\hmwkTitle}{Problem Set\ \#9}
\newcommand{\hmwkDueDate}{Jun 16, 2025}
\newcommand{\hmwkClass}{ECON 121}
\newcommand{\hmwkClassInstructor}{Dr. Jasmine N Fuller}
\newcommand{\hmwkAuthorName}{\textbf{Alejandro Ouslan}}

%
% Title Page
%

\title{
	\vspace{2in}
	\textmd{\textbf{\hmwkClass:\ \hmwkTitle}}\\
	\normalsize\vspace{0.1in}\small{Due\ on\ \hmwkDueDate}\\
	\vspace{0.1in}\large{\textit{\hmwkClassInstructor}}
	\vspace{3in}
}

\author{\hmwkAuthorName}
\date{}

\renewcommand{\part}[1]{\textbf{\large Part \Alph{partCounter}}\stepcounter{partCounter}\\}

%
% Various Helper Commands
%

% Useful for algorithms
\newcommand{\alg}[1]{\textsc{\bfseries \footnotesize #1}}

% For derivatives
\newcommand{\deriv}[1]{\frac{\mathrm{d}}{\mathrm{d}x} (#1)}

% For partial derivatives
\newcommand{\pderiv}[2]{\frac{\partial}{\partial #1} (#2)}

% Integral dx
\newcommand{\dx}{\mathrm{d}x}

% Alias for the Solution section header
\newcommand{\solution}{\textbf{\large Solution}}

% Probability commands: Expectation, Variance, Covariance, Bias
\newcommand{\E}{\mathrm{E}}
\newcommand{\Var}{\mathrm{Var}}
\newcommand{\Cov}{\mathrm{Cov}}
\newcommand{\Bias}{\mathrm{Bias}}

\begin{document}

\maketitle

\pagebreak

\begin{homeworkProblem}
	\begin{enumerate}
		\item Explain why chess in not a static game of complete information.
		\item Consider the following game: I write down a number (1,2, or 3)
		      on a card and set it down. If you guess the right number, I give you
		      \$1. If you guess the wrong number, you give me \$1. Don't solve it; just
		      explain why this is a static game of complete information and draw the matrix
		      representation of the game.
		\item Explain the three criteria we use to evaluate solution concepts, and use them
		      to compare Dominant Strategy Equilibrium with the solution concept of "anything can happen."
		\item What is a Pareto dominated outcome? Compare with a Pareto optimal/efficient outcome.
	\end{enumerate}
\end{homeworkProblem}

\begin{homeworkProblem}
	Consider the following matrix form game:


	\begin{table}[h!]
		\centering
		\begin{tabular}{c|c c c}
			$P_1/P_2$ & X       & Y      & Z       \\
			\hline
			A         & (1, 4)  & (4, 5) & (8, 2)  \\
			B         & (5, 4)  & (1, 6) & (7, 5)  \\
			C         & (0, 11) & (3, 9) & (10, 2) \\
		\end{tabular}
	\end{table}

	Use Iterated Elimination of Dominated Strategies to find the unique solution.
	Show you work.
\end{homeworkProblem}

\begin{homeworkProblem}
	Consider two player: The contestant (C) and the preditor (P). There are two
	boxes, one opaque and one transparent. The transparent box has \$1000 in it for sure.
	$P$ either puts \$0 or \$1,000,000 in the opaque box, unseen by C. Then, C chooses
	either to select only the opaque box (one-box) or both boxes (two-box). C's payoff
	is simply to maximize his monetary gain. P's strange preferences are to demonstrate his
	correct prediction in the following way: Only put money in the box if he thinks
	C will select one box; otherwise, do not put money in the box (ignore mixed strategies for this problem).
	If P predicts correctly, his payoff is 10; otherwise it is 0. Here is the game in matrix form:

	\begin{table}[h!]
		\centering
		\begin{tabular}{c|c c}
			$contestant/predictor$ & One-box        & Two-box              \\
			\hline
			A                      & ($10$, $10^6$) & ($4$, $10^6 + 10^3$) \\
			B                      & ($0$, $0$)     & ($1$, $10^3$)        \\
		\end{tabular}
	\end{table}

	\begin{enumerate}
		\item Explain why this is a static game of complete information.
		\item Find the unique surviving set of strategies using IESD.
		\item Now suppose P (somehow) observe what C will do before C even acts, and this ability
		      is common knowledge. Explain why this not a static game. Who effectively moves first as
		      far as the game is concerned, and who then observes that first move?
		\item If you read ahead to chapter 7 and 8, we will discuss dynamic games of complete information.
		      Solve the dynamic games you described in part (c) using Sub game Perfect Nash Equilibrium.
	\end{enumerate}
\end{homeworkProblem}


\end{document}
