\documentclass{article}

\usepackage{fancyhdr}
\usepackage{extramarks}
\usepackage{amsmath}
\usepackage{amsthm}
\usepackage{amsfonts}
\usepackage{tikz}
\usepackage[plain]{algorithm}
\usepackage{algpseudocode}

\usetikzlibrary{automata,positioning}

%
% Basic Document Settings
%

\topmargin=-0.45in
\evensidemargin=0in
\oddsidemargin=0in
\textwidth=6.5in
\textheight=9.0in
\headsep=0.25in

\linespread{1.1}

\pagestyle{fancy}
\lhead{\hmwkAuthorName}
\chead{\hmwkClass\ (\hmwkClassInstructor): \hmwkTitle}
\rhead{\firstxmark}
\lfoot{\lastxmark}
\cfoot{\thepage}

\renewcommand\headrulewidth{0.4pt}
\renewcommand\footrulewidth{0.4pt}

\setlength\parindent{0pt}

%
% Create Problem Sections
%

\newcommand{\enterProblemHeader}[1]{
	\nobreak\extramarks{}{Problem \arabic{#1} continued on next page\ldots}\nobreak{}
	\nobreak\extramarks{Problem \arabic{#1} (continued)}{Problem \arabic{#1} continued on next page\ldots}\nobreak{}
}

\newcommand{\exitProblemHeader}[1]{
	\nobreak\extramarks{Problem \arabic{#1} (continued)}{Problem \arabic{#1} continued on next page\ldots}\nobreak{}
	\stepcounter{#1}
	\nobreak\extramarks{Problem \arabic{#1}}{}\nobreak{}
}

\setcounter{secnumdepth}{0}
\newcounter{partCounter}
\newcounter{homeworkProblemCounter}
\setcounter{homeworkProblemCounter}{1}
\nobreak\extramarks{Problem \arabic{homeworkProblemCounter}}{}\nobreak{}

%
% Homework Problem Environment
%
% This environment takes an optional argument. When given, it will adjust the
% problem counter. This is useful for when the problems given for your
% assignment aren't sequential. See the last 3 problems of this template for an
% example.
%
\newenvironment{homeworkProblem}[1][-1]{
	\ifnum#1>0
		\setcounter{homeworkProblemCounter}{#1}
	\fi
	\section{Problem \arabic{homeworkProblemCounter}}
	\setcounter{partCounter}{1}
	\enterProblemHeader{homeworkProblemCounter}
}{
	\exitProblemHeader{homeworkProblemCounter}
}

%
% Homework Details
%   - Title
%   - Due date
%   - Class
%   - Section/Time
%   - Instructor
%   - Author
%

\newcommand{\hmwkTitle}{Problem Set\ \#7}
\newcommand{\hmwkDueDate}{Jun 16, 2025}
\newcommand{\hmwkClass}{ECON 121}
\newcommand{\hmwkClassInstructor}{Dr. Jasmine N Fuller}
\newcommand{\hmwkAuthorName}{\textbf{Alejandro Ouslan}}

%
% Title Page
%

\title{
	\vspace{2in}
	\textmd{\textbf{\hmwkClass:\ \hmwkTitle}}\\
	\normalsize\vspace{0.1in}\small{Due\ on\ \hmwkDueDate}\\
	\vspace{0.1in}\large{\textit{\hmwkClassInstructor}}
	\vspace{3in}
}

\author{\hmwkAuthorName}
\date{}

\renewcommand{\part}[1]{\textbf{\large Part \Alph{partCounter}}\stepcounter{partCounter}\\}

%
% Various Helper Commands
%

% Useful for algorithms
\newcommand{\alg}[1]{\textsc{\bfseries \footnotesize #1}}

% For derivatives
\newcommand{\deriv}[1]{\frac{\mathrm{d}}{\mathrm{d}x} (#1)}

% For partial derivatives
\newcommand{\pderiv}[2]{\frac{\partial}{\partial #1} (#2)}

% Integral dx
\newcommand{\dx}{\mathrm{d}x}

% Alias for the Solution section header
\newcommand{\solution}{\textbf{\large Solution}}

% Probability commands: Expectation, Variance, Covariance, Bias
\newcommand{\E}{\mathrm{E}}
\newcommand{\Var}{\mathrm{Var}}
\newcommand{\Cov}{\mathrm{Cov}}
\newcommand{\Bias}{\mathrm{Bias}}

\begin{document}

\maketitle

\pagebreak

\begin{homeworkProblem}
	Consider a competitive producer with a production function of $l^{0.4}k^{0.1}$, labor price
	of $w$ and capital price of $1$ (not $v$, the number one), and an output price of $p$. First suppose
	capital in the short run is fixed at $k_1$.
	\begin{enumerate}
		\item Find the short run cost function.

		      \[
			      f(l, k) = l^{0.4}k^{0.1}
		      \]
		      \[
			      q = l^{0.4}k_1^{0.1}
			      \Rightarrow l^{0.4} = \frac{q}{k_1^{0.1}}
			      \Rightarrow l = \left( \frac{q}{k_1^{0.1}} \right)^{\frac{1}{0.4}} = \left( \frac{q}{k_1^{0.1}} \right)^{2.5}
		      \]
		      \[
			      C(q) = wl + 1 \cdot k_1
		      \]
		      \[
			      C(q) = w \left( \frac{q}{k_1^{0.1}} \right)^{2.5} + k_1
		      \]
		      \[
			      C(q) = w q^{2.5} k_1^{-0.25} + k_1
		      \]

		\item Use the cost function to find the profit maximizing quantity.

		      \[
			      C(q) = w q^{2.5} k_1^{-0.25} + k_1
		      \]
		      \[
			      \pi(q) = pq - C(q) = pq - \left( w q^{2.5} k_1^{-0.25} + k_1 \right)
		      \]
		      \[
			      \frac{d\pi}{dq} = p - \frac{d}{dq} \left( w q^{2.5} k_1^{-0.25} \right) = 0
		      \]
		      \[
			      \frac{d}{dq} \left( w q^{2.5} k_1^{-0.25} \right) = w \cdot 2.5 q^{1.5} k_1^{-0.25}
		      \]
		      \[
			      p = 2.5 w q^{1.5} k_1^{-0.25}
		      \]
		      \[
			      q^{1.5} = \frac{p}{2.5 w k_1^{-0.25}} = \frac{p k_1^{0.25}}{2.5 w}
			      \Rightarrow q = \left( \frac{p k_1^{0.25}}{2.5 w} \right)^{\frac{2}{3}}
		      \]
		\item Find the firm's unconditional demand for labor.
		      \[
			      q = l^{0.4}k_1^{0.1} \Rightarrow l = \left( \frac{q}{k_1^{0.1}} \right)^{\frac{1}{0.4}} = \left( \frac{q}{k_1^{0.1}} \right)^{2.5}
		      \]
		      \[
			      q^* = \left( \frac{p k_1^{0.25}}{2.5 w} \right)^{\frac{2}{3}}
		      \]
		      \[
			      l^* = \left( \frac{1}{k_1^{0.1}} \cdot \left( \frac{p k_1^{0.25}}{2.5 w} \right)^{\frac{2}{3}} \right)^{2.5}
		      \]
		      \[
			      l^* = \left( \left( \frac{p k_1^{0.25}}{2.5 w} \right)^{\frac{2}{3}} \cdot k_1^{-0.1} \right)^{2.5}
			      = \left( \frac{p^{2/3} k_1^{0.25 \cdot \frac{2}{3}}}{(2.5 w)^{2/3} k_1^{0.1}} \right)^{2.5}
		      \]
		      \[
			      = \left( \frac{p^{2/3}}{(2.5 w)^{2/3}} \cdot k_1^{\frac{0.5}{3} - 0.1} \right)^{2.5}
			      = \left( \frac{p^{2/3}}{(2.5 w)^{2/3}} \cdot k_1^{\frac{1}{6} - \frac{1}{10}} \right)^{2.5}
		      \]
		      \[
			      \frac{1}{6} - \frac{1}{10} = \frac{5 - 3}{30} = \frac{2}{30} = \frac{1}{15}
		      \]
		      \[
			      l^* = \left( \frac{p^{2/3}}{(2.5 w)^{2/3}} \cdot k_1^{1/15} \right)^{2.5}
			      = \frac{p^{5/3}}{(2.5 w)^{5/3}} \cdot k_1^{\frac{2.5}{15}} = \frac{p^{5/3}}{(2.5 w)^{5/3}} \cdot k_1^{1/6}
		      \]

		\item Find the profit function.
		      \[
			      q^* = \left( \frac{p k_1^{0.25}}{2.5 w} \right)^{\frac{2}{3}}, \quad
			      C(q^*) = w q^{2.5} k_1^{-0.25} + k_1
		      \]
		      \[
			      pq^* = p \left( \frac{p k_1^{0.25}}{2.5 w} \right)^{\frac{2}{3}} = p^{\frac{5}{3}} \cdot \left( \frac{k_1^{0.25}}{2.5 w} \right)^{\frac{2}{3}}
		      \]
		      \[
			      q^{2.5} = \left( \left( \frac{p k_1^{0.25}}{2.5 w} \right)^{\frac{2}{3}} \right)^{2.5}
			      = \left( \frac{p k_1^{0.25}}{2.5 w} \right)^{\frac{5}{3}}
		      \]
		      \[
			      C(q^*) = w \left( \frac{p k_1^{0.25}}{2.5 w} \right)^{\frac{5}{3}} k_1^{-0.25} + k_1
			      = w \cdot \frac{p^{5/3} k_1^{(0.25)(5/3)}}{(2.5 w)^{5/3}} \cdot k_1^{-0.25} + k_1
		      \]
		      \[
			      = \frac{p^{5/3}}{(2.5)^{5/3} w^{2/3}} \cdot k_1^{\frac{5}{12} - \frac{1}{4}} + k_1
		      \]
		      \[
			      = \frac{p^{5/3}}{(2.5)^{5/3} w^{2/3}} \cdot k_1^{\frac{5}{12} - \frac{3}{12}} + k_1
			      = \frac{p^{5/3}}{(2.5)^{5/3} w^{2/3}} \cdot k_1^{1/6} + k_1
		      \]
		      \[
			      \pi(p, w, k_1) = pq^* - C(q^*)
			      = \frac{p^{5/3}}{(2.5)^{2/3} w^{2/3}} k_1^{1/6}
			      - \left( \frac{p^{5/3}}{(2.5)^{5/3} w^{2/3}} k_1^{1/6} + k_1 \right)
		      \]
		      \[
			      = \left( \frac{p^{5/3}}{(2.5)^{2/3} w^{2/3}} - \frac{p^{5/3}}{(2.5)^{5/3} w^{2/3}} \right) k_1^{1/6} - k_1
		      \]
		      \[
			      = \frac{p^{5/3} k_1^{1/6}}{w^{2/3}} \left( \frac{1}{(2.5)^{2/3}} - \frac{1}{(2.5)^{5/3}} \right) - k_1
		      \]
	\end{enumerate}
	Now suppose capital is variable (long run)
	\begin{enumerate}
		\item Repeat parts (a) though (d). The profit function is extra credit.
		\item Extra credit. Capital remains unconstrained, but the firm has some market power
		      such that $p= 10-2q$. Find the new optimal quantity and unconditional demand for labor.
	\end{enumerate}
\end{homeworkProblem}
\end{document}
