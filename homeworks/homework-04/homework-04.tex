\documentclass{article}

\usepackage{fancyhdr}
\usepackage{extramarks}
\usepackage{amsmath}
\usepackage{amsthm}
\usepackage{amsfonts}
\usepackage{tikz}
\usepackage[plain]{algorithm}
\usepackage{algpseudocode}

\usetikzlibrary{automata,positioning}

%
% Basic Document Settings
%

\topmargin=-0.45in
\evensidemargin=0in
\oddsidemargin=0in
\textwidth=6.5in
\textheight=9.0in
\headsep=0.25in

\linespread{1.1}

\pagestyle{fancy}
\lhead{\hmwkAuthorName}
\chead{\hmwkClass\ (\hmwkClassInstructor): \hmwkTitle}
\rhead{\firstxmark}
\lfoot{\lastxmark}
\cfoot{\thepage}

\renewcommand\headrulewidth{0.4pt}
\renewcommand\footrulewidth{0.4pt}

\setlength\parindent{0pt}

%
% Create Problem Sections
%

\newcommand{\enterProblemHeader}[1]{
	\nobreak\extramarks{}{Problem \arabic{#1} continued on next page\ldots}\nobreak{}
	\nobreak\extramarks{Problem \arabic{#1} (continued)}{Problem \arabic{#1} continued on next page\ldots}\nobreak{}
}

\newcommand{\exitProblemHeader}[1]{
	\nobreak\extramarks{Problem \arabic{#1} (continued)}{Problem \arabic{#1} continued on next page\ldots}\nobreak{}
	\stepcounter{#1}
	\nobreak\extramarks{Problem \arabic{#1}}{}\nobreak{}
}

\setcounter{secnumdepth}{0}
\newcounter{partCounter}
\newcounter{homeworkProblemCounter}
\setcounter{homeworkProblemCounter}{1}
\nobreak\extramarks{Problem \arabic{homeworkProblemCounter}}{}\nobreak{}

%
% Homework Problem Environment
%
% This environment takes an optional argument. When given, it will adjust the
% problem counter. This is useful for when the problems given for your
% assignment aren't sequential. See the last 3 problems of this template for an
% example.
%
\newenvironment{homeworkProblem}[1][-1]{
	\ifnum#1>0
		\setcounter{homeworkProblemCounter}{#1}
	\fi
	\section{Problem \arabic{homeworkProblemCounter}}
	\setcounter{partCounter}{1}
	\enterProblemHeader{homeworkProblemCounter}
}{
	\exitProblemHeader{homeworkProblemCounter}
}

%
% Homework Details
%   - Title
%   - Due date
%   - Class
%   - Section/Time
%   - Instructor
%   - Author
%

\newcommand{\hmwkTitle}{Problem Set\ \#4}
\newcommand{\hmwkDueDate}{May 29, 2025}
\newcommand{\hmwkClass}{ECON 219}
\newcommand{\hmwkClassInstructor}{Dr. Jasmine N Fuller}
\newcommand{\hmwkAuthorName}{\textbf{Alejandro Ouslan}}

%
% Title Page
%

\title{
	\vspace{2in}
	\textmd{\textbf{\hmwkClass:\ \hmwkTitle}}\\
	\normalsize\vspace{0.1in}\small{Due\ on\ \hmwkDueDate}\\
	\vspace{0.1in}\large{\textit{\hmwkClassInstructor}}
	\vspace{3in}
}

\author{\hmwkAuthorName}
\date{}

\renewcommand{\part}[1]{\textbf{\large Part \Alph{partCounter}}\stepcounter{partCounter}\\}

%
% Various Helper Commands
%

% Useful for algorithms
\newcommand{\alg}[1]{\textsc{\bfseries \footnotesize #1}}

% For derivatives
\newcommand{\deriv}[1]{\frac{\mathrm{d}}{\mathrm{d}x} (#1)}

% For partial derivatives
\newcommand{\pderiv}[2]{\frac{\partial}{\partial #1} (#2)}

% Integral dx
\newcommand{\dx}{\mathrm{d}x}

% Alias for the Solution section header
\newcommand{\solution}{\textbf{\large Solution}}

% Probability commands: Expectation, Variance, Covariance, Bias
\newcommand{\E}{\mathrm{E}}
\newcommand{\Var}{\mathrm{Var}}
\newcommand{\Cov}{\mathrm{Cov}}
\newcommand{\Bias}{\mathrm{Bias}}

\begin{document}

\maketitle

\pagebreak

% Problem 9.5
\begin{homeworkProblem}
	As we have seen in many places, the general Cobb-Douglas production function for two
	inputs is given by
	$$q=f(k,l)=AK^{\alpha} l^{\beta}$$
	where $0<\alpha<1$ and $0<\beta<1$. For this production function:
	\begin{enumerate}
		\item Show that $f_k>0$, $f_1 > 0$, $f_{kk}<0$, $f_{ll}<0$, and $f_{kl}>0$.
		      \begin{align*}
			      f_k    & = \frac{\partial f}{\partial k} = A \alpha k^{\alpha - 1} l^{\beta} > 0                        \\
			      f_l    & = \frac{\partial f}{\partial l} = A \beta k^{\alpha} l^{\beta - 1} > 0                         \\
			      f_{kk} & = \frac{\partial^2 f}{\partial k^2} = A \alpha (\alpha - 1) k^{\alpha - 2} l^{\beta} < 0       \\
			      f_{ll} & = \frac{\partial^2 f}{\partial l^2} = A \beta (\beta - 1) k^{\alpha} l^{\beta - 2} < 0         \\
			      f_{kl} & = \frac{\partial^2 f}{\partial k \partial l} = A \alpha \beta k^{\alpha - 1} l^{\beta - 1} > 0
		      \end{align*}
		\item Show that $e_{q,k} = \alpha$ and $e_{q,l}=\beta$.
		      \begin{align*}
			      e_{q,k} & = \frac{\partial f}{\partial k} \cdot \frac{k}{f(k,l)} = \frac{A \alpha k^{\alpha - 1} l^{\beta} \cdot k}{A k^{\alpha} l^{\beta}} = \alpha \\
			      e_{q,l} & = \frac{\partial f}{\partial l} \cdot \frac{l}{f(k,l)} = \frac{A \beta k^{\alpha} l^{\beta - 1} \cdot l}{A k^{\alpha} l^{\beta}} = \beta
		      \end{align*}
		\item In footnote 5, we defined the scale elasticity as
		      $$e_{q,t}=\frac{\partial f(tk,tl)}{\partial t} \cdot \frac{t}{f(tk,tl)}$$
		      where the expression is to be evaluated at $t=1$. Show that, for this Cobb-Douglas function,
		      $e_{q,t} = \alpha + \beta$/ Hence in this case the elasticity and the returns to scale of the production
		      function agree (for more on this concept see Problem 9.9).
		      \begin{align*}
			      \frac{dF}{dt} & = A (\alpha + \beta) t^{\alpha + \beta - 1} k^{\alpha} l^{\beta} \\
			      f(tk, tl)     & = A t^{\alpha + \beta} k^{\alpha} l^{\beta}
		      \end{align*}
		      \[
			      e_{q,t} = \left. \frac{\partial f(tk,tl)}{\partial t} \cdot \frac{t}{f(tk,tl)} \right|_{t=1}
			      = \left. \frac{A (\alpha + \beta) t^{\alpha + \beta - 1} k^{\alpha} l^{\beta} \cdot t}{A t^{\alpha + \beta} k^{\alpha} l^{\beta}} \right|_{t=1} = \alpha + \beta
		      \]
		\item Show that this function is quasi-concave.


		      \[
			      \log f(k, l) = \log A + \alpha \log k + \beta \log l
		      \]

		      \textbf{Answer:} This function is concave in \( \log k \) and \( \log l \), which implies \( f \) is log-concave, and hence quasi-concave.
		\item Show that the function is concave for $\alpha + \beta \ge 1$ but not concave for $alpha + \beta > 1$.

		      \[
			      H =
			      \begin{bmatrix}
				      f_{kk} & f_{kl} \\
				      f_{lk} & f_{ll}
			      \end{bmatrix}
			      =
			      A
			      \begin{bmatrix}
				      \alpha(\alpha - 1) k^{\alpha - 2} l^{\beta} & \alpha \beta k^{\alpha - 1} l^{\beta - 1} \\
				      \alpha \beta k^{\alpha - 1} l^{\beta - 1}   & \beta(\beta - 1) k^{\alpha} l^{\beta - 2}
			      \end{bmatrix}
		      \]


		      \[
			      \det(H) = A^2 k^{2\alpha - 2} l^{2\beta - 2} \left[(\alpha - 1)(\beta - 1) - \alpha \beta \right]
		      \]

		      \[
			      (\alpha - 1)(\beta - 1) \geq \alpha \beta \quad \Rightarrow \quad \alpha + \beta \leq 1
		      \]

		      \textbf{Answer:} Thus, the function is concave when \( \alpha + \beta \leq 1 \), and not concave when \( \alpha + \beta > 1 \).
	\end{enumerate}
\end{homeworkProblem}

% Problem 9.7
\begin{homeworkProblem}
	Consider a generalization of the production function in Example 9.3:
	$$q= \beta_0 + \beta_1 \sqrt{kl} + \beta_2 k + \beta_3 l$$
	where
	$$0 \ge \beta_i \quad i=0,\ldots,3$$
	\begin{enumerate}
		\item if this function is to exhibit constant returns to scale, what restrictions should be placed on the parameters
		      $\beta_0, \cdots, \beta_3$?
		      \[
			      q(tk, tl) = \beta_0 + \beta_1 \sqrt{tk \cdot tl} + \beta_2 (tk) + \beta_3 (tl) = \beta_0 + \beta_1 t \sqrt{kl} + \beta_2 tk + \beta_3 tl
		      \]
		      \[
			      t q(k, l) = t(\beta_0 + \beta_1 \sqrt{kl} + \beta_2 k + \beta_3 l) = t\beta_0 + t\beta_1 \sqrt{kl} + t\beta_2 k + t\beta_3 l
		      \]
		      Equality holds if and only if \(\beta_0 = 0\). Hence, the condition for CRS is:
		      \[
			      \boxed{\beta_0 = 0}
		      \]
		\item Show that, in the constant returns-to-scale case, this function exhibits diminishing marginal
		      productivities and that marginal productivity function ar homogeneous of degree 0.

		      \[
			      q = \beta_1 \sqrt{kl} + \beta_2 k + \beta_3 l
		      \]
		      \[
			      \frac{\partial q}{\partial k} = \frac{\beta_1}{2} \sqrt{\frac{l}{k}} + \beta_2, \qquad
			      \frac{\partial q}{\partial l} = \frac{\beta_1}{2} \sqrt{\frac{k}{l}} + \beta_3
		      \]
		      \[
			      \frac{\partial^2 q}{\partial k^2} = -\frac{\beta_1}{4} \sqrt{\frac{l}{k^3}} < 0, \qquad
			      \frac{\partial^2 q}{\partial l^2} = -\frac{\beta_1}{4} \sqrt{\frac{k}{l^3}} < 0
		      \]
		      \[
			      \frac{\partial q(tk, tl)}{\partial (tk)} = \frac{\beta_1}{2} \sqrt{\frac{tl}{tk}} + \beta_2 = \frac{\beta_1}{2} \sqrt{\frac{l}{k}} + \beta_2 = \frac{\partial q}{\partial k}
		      \]
		      \textbf{Answer:} Marginal product functions are homogeneous of degree 0
		\item Calculate $\sigma$ in this case. Although $\sigma$ is not in general constant, for what values of the $\beta$'s does
		      $\sigma=0,1,\infty$

		      \[
			      \text{MRTS} = \frac{\partial q/\partial k}{\partial q/\partial l} = \frac{\frac{\beta_1}{2} \sqrt{l/k} + \beta_2}{\frac{\beta_1}{2} \sqrt{k/l} + \beta_3}
		      \]

		      cases:
		      \begin{itemize}
			      \item If \(\beta_1 = 0\), then \(q = \beta_2 k + \beta_3 l\) (perfect substitutes): \(\sigma = \infty\)
			      \item If \(\beta_2 = \beta_3 = 0\), then \(q = \beta_1 \sqrt{kl}\) (Cobb-Douglas): \(\sigma = 1\)
			      \item If \(\beta_1 = \beta_2 = 0\) or \(\beta_1 = \beta_3 = 0\): one input is essential, no substitutability: \(\sigma = 0\)
		      \end{itemize}

		      \[
			      \begin{aligned}
				      \sigma = \infty & \quad \text{if } \beta_1 = 0                                    \\
				      \sigma = 1      & \quad \text{if } \beta_2 = \beta_3 = 0                          \\
				      \sigma = 0      & \quad \text{if } \beta_1 = 0 \text{ and only one input matters}
			      \end{aligned}
		      \]
	\end{enumerate}

\end{homeworkProblem}

% Additional Application
\begin{homeworkProblem}
	Consider the production function $q= f(k,l) = (\sqrt{k} + \sqrt{l})^2$. The cost of labor and capital
	are $w$ and  $v$ respectively. Aside from the cost of labor and capital, there are no cost.
	\begin{enumerate}
		\item What type of production function is this? Hint: Not Cobb-Douglas

		      special case of Constant Elasticity of Substitution production function.
		\item Use your answer from part (a) to find the elasticity of substitution. Hint: There is a simple formula for
		      \[
			      \sigma = \frac{1}{1 - r}
		      \]
		      \( r = \frac{1}{2} \Rightarrow \sigma = \frac{1}{1 - \frac{1}{2}} = 2 \). \\
		      \[
			      \boxed{\sigma = 2}
		      \]
		\item Find RTS (l for k).
		      \[
			      MP_k = \frac{\partial q}{\partial k} = \frac{\sqrt{k} + \sqrt{l}}{\sqrt{k}}, \quad MP_l = \frac{\partial q}{\partial l} = \frac{\sqrt{k} + \sqrt{l}}{\sqrt{l}}
		      \]
		      \[
			      RTS_{l,k} = \frac{MP_l}{MP_k} = \frac{\sqrt{k}}{\sqrt{l}} = \boxed{\sqrt{\frac{k}{l}}}
		      \]
		\item Find the total cost function.
		      \[
			      C = v x^2 + w (\sqrt{q} - x)^2 = (v + w)x^2 - 2w x \sqrt{q} + wq
		      \]
		      \[
			      \frac{dC}{dx} = 2(v + w)x - 2w \sqrt{q} = 0 \Rightarrow x = \frac{w}{v + w} \sqrt{q}
		      \]
		      \[
			      k = \left( \frac{w}{v + w} \right)^2 q, \quad l = \left( \frac{v}{v + w} \right)^2 q
		      \]
		      \[
			      C(q) = v k + w l = \frac{vw}{v + w} q
		      \]
		      \[
			      C(q) = \frac{vw}{v + w} q
		      \]
		\item Use your answer from part (d) to find the average cost.
		      \[
			      AC = \frac{C(q)}{q} = \boxed{\frac{vw}{v + w}}
		      \]
		\item Use your answer from part (d) to find marginal cost.
		      \[
			      MC = \frac{dC}{dq} = \boxed{\frac{vw}{v + w}}
		      \]

		\item Is the production function homothetic?
		      \[
			      f(k, l) = (\sqrt{k} + \sqrt{l})^2 \quad \text{is a monotonic transformation of} \quad \sqrt{k} + \sqrt{l}
		      \]
		      \boxed{\text{Yes, it is homothetic.}}
		\item Does the production function exhibit increasing, constant, or decreasing return to scale?
		      \[
			      f(\lambda k, \lambda l) = (\sqrt{\lambda k} + \sqrt{\lambda l})^2 = \lambda(\sqrt{k} + \sqrt{l})^2 = \lambda f(k, l)
		      \]
		      \boxed{\text{Constant Returns to Scale}}

		\item Is the expansion path linear?
		      \[
			      \frac{k}{l} = \left( \frac{w}{v} \right)^2 \Rightarrow \boxed{\text{Yes,linear.}}
		      \]
		\item In general, what can you say about AC and MC for homothetic production functions with no fixed cost?
		      For homothetic production functions with no fixed cost:
		      \begin{itemize}
			      \item If the production function has constant returns to scale, then both AC and MC are constant and equal.
			      \item This is because total cost is linear in output.
		      \end{itemize}
		      \[
			      \boxed{\text{AC and MC are constant and equal for CRS, homothetic functions with no fixed cost.}}
		      \]
	\end{enumerate}
\end{homeworkProblem}
\end{document}
