\documentclass{article}

\usepackage{fancyhdr}
\usepackage{extramarks}
\usepackage{amsmath}
\usepackage{amsthm}
\usepackage{amsfonts}
\usepackage{tikz}
\usepackage[plain]{algorithm}
\usepackage{algpseudocode}

\usetikzlibrary{automata,positioning}

%
% Basic Document Settings
%

\topmargin=-0.45in
\evensidemargin=0in
\oddsidemargin=0in
\textwidth=6.5in
\textheight=9.0in
\headsep=0.25in

\linespread{1.1}

\pagestyle{fancy}
\lhead{\hmwkAuthorName}
\chead{\hmwkClass\ (\hmwkClassInstructor): \hmwkTitle}
\rhead{\firstxmark}
\lfoot{\lastxmark}
\cfoot{\thepage}

\renewcommand\headrulewidth{0.4pt}
\renewcommand\footrulewidth{0.4pt}

\setlength\parindent{0pt}

%
% Create Problem Sections
%

\newcommand{\enterProblemHeader}[1]{
	\nobreak\extramarks{}{Problem \arabic{#1} continued on next page\ldots}\nobreak{}
	\nobreak\extramarks{Problem \arabic{#1} (continued)}{Problem \arabic{#1} continued on next page\ldots}\nobreak{}
}

\newcommand{\exitProblemHeader}[1]{
	\nobreak\extramarks{Problem \arabic{#1} (continued)}{Problem \arabic{#1} continued on next page\ldots}\nobreak{}
	\stepcounter{#1}
	\nobreak\extramarks{Problem \arabic{#1}}{}\nobreak{}
}

\setcounter{secnumdepth}{0}
\newcounter{partCounter}
\newcounter{homeworkProblemCounter}
\setcounter{homeworkProblemCounter}{1}
\nobreak\extramarks{Problem \arabic{homeworkProblemCounter}}{}\nobreak{}

%
% Homework Problem Environment
%
% This environment takes an optional argument. When given, it will adjust the
% problem counter. This is useful for when the problems given for your
% assignment aren't sequential. See the last 3 problems of this template for an
% example.
%
\newenvironment{homeworkProblem}[1][-1]{
	\ifnum#1>0
		\setcounter{homeworkProblemCounter}{#1}
	\fi
	\section{Problem \arabic{homeworkProblemCounter}}
	\setcounter{partCounter}{1}
	\enterProblemHeader{homeworkProblemCounter}
}{
	\exitProblemHeader{homeworkProblemCounter}
}

%
% Homework Details
%   - Title
%   - Due date
%   - Class
%   - Section/Time
%   - Instructor
%   - Author
%

\newcommand{\hmwkTitle}{Midterm\ \#1}
\newcommand{\hmwkDueDate}{May 29, 2025}
\newcommand{\hmwkClass}{ECON 219}
\newcommand{\hmwkClassInstructor}{Dr. Jasmine N Fuller}
\newcommand{\hmwkAuthorName}{\textbf{Alejandro Ouslan}}

%
% Title Page
%

\title{
	\vspace{2in}
	\textmd{\textbf{\hmwkClass:\ \hmwkTitle}}\\
	\normalsize\vspace{0.1in}\small{Due\ on\ \hmwkDueDate}\\
	\vspace{0.1in}\large{\textit{\hmwkClassInstructor}}
	\vspace{3in}
}

\author{\hmwkAuthorName}
\date{}

\renewcommand{\part}[1]{\textbf{\large Part \Alph{partCounter}}\stepcounter{partCounter}\\}

%
% Various Helper Commands
%

% Useful for algorithms
\newcommand{\alg}[1]{\textsc{\bfseries \footnotesize #1}}

% For derivatives
\newcommand{\deriv}[1]{\frac{\mathrm{d}}{\mathrm{d}x} (#1)}

% For partial derivatives
\newcommand{\pderiv}[2]{\frac{\partial}{\partial #1} (#2)}

% Integral dx
\newcommand{\dx}{\mathrm{d}x}

% Alias for the Solution section header
\newcommand{\solution}{\textbf{\large Solution}}

% Probability commands: Expectation, Variance, Covariance, Bias
\newcommand{\E}{\mathrm{E}}
\newcommand{\Var}{\mathrm{Var}}
\newcommand{\Cov}{\mathrm{Cov}}
\newcommand{\Bias}{\mathrm{Bias}}

\begin{document}

\maketitle

\pagebreak

% Problem 1
\begin{homeworkProblem}
	Let $U(x,y) = \sqrt{x} + \sqrt{y}$. Find the following:
	\begin{enumerate}
		\item Marshallian demand function for $x$ and $y$.
		\item Indirect utility function.
		\item Expenditure function.
		\item Write the Slutsky equation for good $x$.
	\end{enumerate}
\end{homeworkProblem}

% Problem 2 
\begin{homeworkProblem}
	Determine the marginal rate of substitution (MRS) for the following utility functions:
	\begin{enumerate}
		\item $U(x,y) = X^{1/4} + 3y$
		\item $U(x,y) = X^{1/4} + y^{1/4}$
	\end{enumerate}
\end{homeworkProblem}

% Problem 3
\begin{homeworkProblem}
	What si the effect of imposing a \$10,000 lump-sum tax on a monopolist? Assume the firm
	continues to operate after the tax.
	\begin{enumerate}
		\item The firm experiences no change in the profit-maximizing price or quantity or
		      in its profit.
		\item The firm experiences no change in the profit-maximizing price and quantity,
		      but its profit decreases
		\item The firm experiences no change in the profit-maximizing price and quantity,
		      but its profit decreases
		\item The firm's profit-maximizing quantity increases, but price does not change.
		\item The firm's profit-maximizing price decreases, but quantity does not change.
	\end{enumerate}
\end{homeworkProblem}

% Problem 4 
\begin{homeworkProblem}
	Consider the production function $q = f(k,l) = l^{0.5}k^{0.1}$. The cost of labor and
	capital are $w$ and $v$ respectively. Aside from the cost of labor and capital, there
	there are no cost. Initially assume $k$ is fixed at $k_1$.
	\begin{enumerate}
		\item Is this function constant return to scale, increasing returns to scale, or decreasing
		      returns to scale? Explain.
		\item Find the short run cost function $C_{SR(q)}$
		\item Find short run marginal cost.
		\item Find short run average cost.
		\item Demostrare that cost is minimized when $AC = MC$.

		      Now assume all inputs are variables.

		\item Find the long run cost function $C_{LR(q)}$.
		\item Find long run marginal cost.
		\item Use you answer from part (a) to find the elasticity of substitution
	\end{enumerate}
\end{homeworkProblem}

% Problem 6
\begin{homeworkProblem}
	A monopolist has a cost function $C(q)= \frac{q^2}{2}$. The monopolist faces a
	market demand curve given $q=10-p$.
	\begin{enumerate}
		\item Find marginal cost.
		\item Calculate the profit-maximizing price-quantity combination for the monopolist.
		      Also calculate the monopolist's profits.
		\item What output level would be produced by this industry under perfect competition?
		\item Calculate the consumer surplus obtained in case (c). Show that this exceeds
		      the sum of monopolist's profit and the consumer surplus received in case (b)
		      What is the value of the dead weight loss from the monopolization?
	\end{enumerate}
\end{homeworkProblem}

% Prroblem 7
\begin{homeworkProblem}
	If the value of th eprice elasticity of demand is 0.2, this means that
	\begin{enumerate}
		\item a 20\% decreases in price cause a 1\% increase in quantity demanded.
		\item a 5\% decrease in price causes a 1\% increase in quantity demanded
		\item a 0.2\% decrease in price causes a 0.2\% increase in quantity demanded.
		\item a 100\% decrease in price cause a 200\% increase in quantity demanded.
	\end{enumerate}
\end{homeworkProblem}

% Problem 8
\begin{homeworkProblem}
	Suppose the production possibility frontier for an economy that produces on public
	and one private good (x) is given by
	$$x^2 + 100 y^2 = 5000$$
	This economy is populated by 100 identical individuals, each with a utility function
	of the from
	$$U(x,y) = \sqrt{xy}$$
	Where $x_i$ is the individual's share of private good production (=x/100).
	Notice that the public good is nonexclusive and that everyone benefits equally
	from its level of production.
	\begin{enumerate}
		\item If the market for $x$ and $y$ were perfectly competitive, what levels of those goods
		      would be produced? What would the typical individual's utlitty be in this situation?
		\item What are the optimal production levels for $x$ and $y$? What would the typical
		      individual's utility level be? (Hint: The numbers in this problem do not come out evenly, and some approximations should suffice.)
	\end{enumerate}
\end{homeworkProblem}

% Problem 9
\begin{homeworkProblem}
	Which of the following is true of perfect price discrimination?
	\begin{enumerate}
		\item Profit is lower than it would be without discrimination.
		\item Revenue is higher then it would be without discrimination.
		\item (Hint: The numbers in this problem do not come out evenly,
		      and some approximations should suffice.)
		\item Profit is zero
	\end{enumerate}
\end{homeworkProblem}

% Problem 10
\begin{homeworkProblem}
	If fixed cost at $Q=100$ is \$130, then
	\begin{enumerate}
		\item fixed cost $Q = 0$ is \$0
		\item fixed cost at Q = 0 is less than \$130.
		\item fixed cost at Q = 200 is \$260.
		\item fixed cost at Q = 200 is \$130.
		\item it is impossible to calculate fixed costs at any other quantity.
	\end{enumerate}
\end{homeworkProblem}


\end{document}
