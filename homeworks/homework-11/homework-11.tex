\documentclass{article}

\usepackage{fancyhdr}
\usepackage{extramarks}
\usepackage{amsmath}
\usepackage{amsthm}
\usepackage{amsfonts}
\usepackage{tikz}
\usepackage{istgame}
\usepackage[plain]{algorithm}
\usepackage[inline,shortlabels]{enumitem}
\usepackage{algpseudocode}

\usetikzlibrary{automata,positioning}

%
% Basic Document Settings
%

\topmargin=-0.45in
\evensidemargin=0in
\oddsidemargin=0in
\textwidth=6.5in
\textheight=9.0in
\headsep=0.25in

\linespread{1.1}

\pagestyle{fancy}
\lhead{\hmwkAuthorName}
\chead{\hmwkClass\ (\hmwkClassInstructor): \hmwkTitle}
\rhead{\firstxmark}
\lfoot{\lastxmark}
\cfoot{\thepage}

\renewcommand\headrulewidth{0.4pt}
\renewcommand\footrulewidth{0.4pt}

\setlength\parindent{0pt}

%
% Create Problem Sections
%

\newcommand{\enterProblemHeader}[1]{
	\nobreak\extramarks{}{Problem \arabic{#1} continued on next page\ldots}\nobreak{}
	\nobreak\extramarks{Problem \arabic{#1} (continued)}{Problem \arabic{#1} continued on next page\ldots}\nobreak{}
}

\newcommand{\exitProblemHeader}[1]{
	\nobreak\extramarks{Problem \arabic{#1} (continued)}{Problem \arabic{#1} continued on next page\ldots}\nobreak{}
	\stepcounter{#1}
	\nobreak\extramarks{Problem \arabic{#1}}{}\nobreak{}
}

\setcounter{secnumdepth}{0}
\newcounter{partCounter}
\newcounter{homeworkProblemCounter}
\setcounter{homeworkProblemCounter}{1}
\nobreak\extramarks{Problem \arabic{homeworkProblemCounter}}{}\nobreak{}

%
% Homework Problem Environment
%
% This environment takes an optional argument. When given, it will adjust the
% problem counter. This is useful for when the problems given for your
% assignment aren't sequential. See the last 3 problems of this template for an
% example.
%
\newenvironment{homeworkProblem}[1][-1]{
	\ifnum#1>0
		\setcounter{homeworkProblemCounter}{#1}
	\fi
	\section{Problem \arabic{homeworkProblemCounter}}
	\setcounter{partCounter}{1}
	\enterProblemHeader{homeworkProblemCounter}
}{
	\exitProblemHeader{homeworkProblemCounter}
}

%
% Homework Details
%   - Title
%   - Due date
%   - Class
%   - Section/Time
%   - Instructor
%   - Author
%

\newcommand{\hmwkTitle}{Problem Set\ \#11}
\newcommand{\hmwkDueDate}{Jun 16, 2025}
\newcommand{\hmwkClass}{ECON 121}
\newcommand{\hmwkClassInstructor}{Dr. Jasmine N Fuller}
\newcommand{\hmwkAuthorName}{\textbf{Alejandro Ouslan}}

%
% Title Page
%

\title{
	\vspace{2in}
	\textmd{\textbf{\hmwkClass:\ \hmwkTitle}}\\
	\normalsize\vspace{0.1in}\small{Due\ on\ \hmwkDueDate}\\
	\vspace{0.1in}\large{\textit{\hmwkClassInstructor}}
	\vspace{3in}
}

\author{\hmwkAuthorName}
\date{}

\renewcommand{\part}[1]{\textbf{\large Part \Alph{partCounter}}\stepcounter{partCounter}\\}

%
% Various Helper Commands
%

% Useful for algorithms
\newcommand{\alg}[1]{\textsc{\bfseries \footnotesize #1}}

% For derivatives
\newcommand{\deriv}[1]{\frac{\mathrm{d}}{\mathrm{d}x} (#1)}

% For partial derivatives
\newcommand{\pderiv}[2]{\frac{\partial}{\partial #1} (#2)}

% Integral dx
\newcommand{\dx}{\mathrm{d}x}

% Alias for the Solution section header
\newcommand{\solution}{\textbf{\large Solution}}

% Probability commands: Expectation, Variance, Covariance, Bias
\newcommand{\E}{\mathrm{E}}
\newcommand{\Var}{\mathrm{Var}}
\newcommand{\Cov}{\mathrm{Cov}}
\newcommand{\Bias}{\mathrm{Bias}}

\begin{document}

\maketitle

\pagebreak

% 7.4
\begin{homeworkProblem}
	Centipedes: Imagine a two player game that proceeds as follows. A pot of
	money is created with \$6 in it initially. Player 1 moves first, then player 2,
	then player 1 again and finally player 2 again. At each player’s turn to move,
	he has two possible actions: grab (G) or share (S). If he grabs, he gets 23 of
	the current pot of money, the other player gets 1/3 of the pot and the game
	ends. If he shares then the size of the current pot is multiplied by 3/2 and the
	next player gets to move. At the last stage in which player 2 moves, if he
	chooses share then the pot is still multiplied by 3/2 , player 2 gets 1/3 of the pot
	and player 1 gets 2/3 of the pot.

	\begin{enumerate}[(a)]
		\item Model this as an extensive form game tree. Is it a game of perfext
		      or imperfect information?
		\item How many terminal nodes does the game have? How many information sets?
		\item How many pure stategies does each player have?
		\item Find the Nash equilibria of this game. How many outcomes can be
		      supported in equilibrium?
		\item Now imagin that at the last stage in which player 2 move, if he chooses
		      to share then the pot is qually split among the players.
		      Does your answer to part (d) above change?
	\end{enumerate}

\end{homeworkProblem}

% 8.6
\begin{homeworkProblem}
	Consider two firms that play a Cournot competion game with demand $p = 100 -q$,
	And costs for each firm given by $c_i(q_i) = 10 q_i$. Imagine that before the
	two firms play the cournot game, firm 1 can invest in cost reduction. If it invests, the cost of firms 1 will drop to
	$c_1(q_1) = 5q_1$. The cost of investment is $F > 0$. Firm 2 does not have this investment oppertunity.

\end{homeworkProblem}

% 8.16
\begin{homeworkProblem}
\end{homeworkProblem}

% application 1
\begin{homeworkProblem}
	Someone stole from a bloodthirsty crime boss. The boss is questioning one of his workers,
	seeking a confession. He says to the worker, “If you do not confess your crime, I will kill
	you.” The worker agrees that the boss is going to kill him, but he notes that the boss’ threat
	is not credible. Why is the worker correct? (Hint: What happens if the worker confesses?)
\end{homeworkProblem}

% application 2
\begin{homeworkProblem}
	\begin{center}
		\begin{istgame}
			\xtdistance{15mm}{30mm}
			\istroot(0){1}
			\istb{A}[above left]{(1,0)}
			\istb{B}[above right]
			\endist
			\istroot(1)(0-2)<30>{2}
			\istb{C}[al]{(0,2)}
			\istb{D}[ar]
			\endist
			\istroot(2)(1-2)<30>{1}
			\istb{E}[al]{(3,0)}
			\istb{F}[ar]
			\endist
			\istroot(3)(2-2)<30>{2}
			\istb{G}[al]{(0,4)}
			\istb{H}[ar]{(0,5)}
			\endist
		\end{istgame}
	\end{center}

	\begin{center}

		\begin{istgame}
			% Increase horizontal spacing
			\xtdistance{15mm}{60mm}

			% Root node
			\istroot(0){1}
			\istb{A}[al]
			\istb{B}[ar]
			\endist
			\xtdistance{10mm}{30mm}

			% Subtree after A (shifted slightly to the left for clarity)
			\istroot(4)(0-1)<above left>{2}
			\istb{C}[al]
			\istb{D}[ar]
			\endist

			% Subtree after D (shifted to the right using extra horizontal distance)
			\istroot(1)(0-2)<above right>{2}
			\istb{E}[al]
			\istb{F}[ar]
			\endist

			\xtdistance{20mm}{20mm}

			\istroot(5)(4-1)<above left>
			\istb{G}[al]{(1,2)}
			\istb{H}[ar]{(7,8)}
			\endist
			\istroot(6)(4-2)<above right>
			\istb{G}[al]{(8,7)}
			\istb{H}[ar]{(2,1)}
			\endist

			\istroot(2)(1-1)<above left>
			\istb{I}[al]{(3,4)}
			\istb{J}[ar]{(5,6)}
			\endist
			\istroot(3)(1-2)<above right>
			\istb{I}[al]{(6,5)}
			\istb{J}[ar]{(4,3)}
			\endist


			\xtInfoset(6)(5){1}
			\xtInfoset(2)(3){1}

		\end{istgame}
	\end{center}
\end{homeworkProblem}

% application 3
\begin{homeworkProblem}
	Consider a simplified version of the Centipedes game. There is a stack of N dollar bills.
	Player 1 can take \$2 and end the game, or she can take one dollar and allow Player 2 to act.
	Then, if there are at least \$2 left, Player 2 can either take \$2 and end the game or take \$1
	and allow Player 1 to act, and so on. If there is only \$1 left, the acting player takes the
	dollar.

	\begin{enumerate}[(a)]
		\item What is the SPINE for $N=2$? $N=3$?
		\item is the outcoe of the SPINE for $N=10000000$
		\item A complete strategy is "always take \$1 no matter what." What is the best response
		      to that strategy, for some $N$?
		\item With words, while two chess grandmasters will always end up taking \$2 immediately.
	\end{enumerate}

\end{homeworkProblem}



\end{document}

